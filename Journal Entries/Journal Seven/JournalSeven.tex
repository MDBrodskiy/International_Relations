%%%%%%%%%%%%%%%%%%%%%%%%%%%%%%%%%%%%%%%%%%%%%%%%%%%%%%%%%%%%%%%%%%%%%%%%%%%%%%%%%%%%%%%%%%%%%%%%%%%%%%%%%%%%%%%%%%%%%%%%%%%%%%%%%%%%%%%%%%%%%%%%%%%%%%%%%%%%%%%%%%%%%%%%%%%%%%%%%%%%%%%%%%%%
% Written By Michael Brodskiy
% Class: International Relations
% Professor: J. Kropf
%%%%%%%%%%%%%%%%%%%%%%%%%%%%%%%%%%%%%%%%%%%%%%%%%%%%%%%%%%%%%%%%%%%%%%%%%%%%%%%%%%%%%%%%%%%%%%%%%%%%%%%%%%%%%%%%%%%%%%%%%%%%%%%%%%%%%%%%%%%%%%%%%%%%%%%%%%%%%%%%%%%%%%%%%%%%%%%%%%%%%%%%%%%%

\documentclass[12pt]{article} 
\usepackage{alphalph}
\usepackage[utf8]{inputenc}
\usepackage[russian,english]{babel}
\usepackage{titling}
\usepackage{amsmath}
\usepackage{graphicx}
\usepackage{enumitem}
\usepackage{amssymb}
\usepackage[super]{nth}
\usepackage{everysel}
\usepackage{ragged2e}
\usepackage{geometry}
\usepackage{fancyhdr}
\usepackage{cancel}
\usepackage{siunitx}
\usepackage{xcolor}
\usepackage{setspace}
\doublespacing
\geometry{top=1.0in,bottom=1.0in,left=1.0in,right=1.0in}
\newcommand{\subtitle}[1]{%
  \posttitle{%
    \par\end{center}
    \begin{center}\large#1\end{center}
    \vskip0.5em}%

}
\usepackage{hyperref}
\hypersetup{
colorlinks=true,
linkcolor=blue,
filecolor=magenta,      
urlcolor=blue,
citecolor=blue,
}

\urlstyle{same}


\title{Journal One}
\date{June 15, 2021}
\author{Michael Brodskiy\\ \small Instructor: Prof. Kropf}

% Mathematical Operations:

% Sum: $$\sum_{n=a}^{b} f(x) $$
% Integral: $$\int_{lower}^{upper} f(x) dx$$
% Limit: $$\lim_{x\to\infty} f(x)$$

\begin{document}

\flushleft Michael Brodskiy

\begin{center}

 \underline{Journal Seven}

\end{center}

\begin{justify}
  \paragraph{} Laurence Shoup's argument that the trajectory of U.S.\ foreign policy is predominantly controlled by the Council on Foreign Relations falls under the neo-realist theory\footnote{although Shoup frequents the incorporation of Marxist (i.e.\ classist) ideology, his fixation seems to be match the tone of his anti-Trumpist baseless remarks.} of international relations and the implications it has over the security dilemma. Although Shoup begins his discourse with baseless allegations he makes an interesting point suggesting that changing presidential incumbents does not necessarily change the trajectory of foreign policy. The Council on Foreign Relations is comprised of senior U.S.\ policy makers, career intelligence officers, and intellectual elitists; collectively these individuals understand the necessity to maintain official channels supportive of bilaterally constructive dialogue and therefore really \emph{drive} international relations outside the scope of the Oval Office. Consider for example George H.W.\ Bush and his successor William Clinton, domestically the two politicians represented opposing political parties, however on the international stage their positions was not at all discrepant (i.e.\ Bosnia and the general U.S.\ position throughout the Yugoslavian crisis). Shoup's critique of American plutocracy is inadvertently specious — for obvious reasons American diplomats are, more often than not, matriculants of the highest esteemed universities and higher education establishments. It is paramount for the foreign service apparatus to employ the brightest and most capable as representatives of American diplomacy; to simply dismiss this convention as a plutocracy is an exemplification of misunderstanding the concept of international relations of the highest magnitude. While Shoup is absolutely correct in suggesting that Biden’s foreign policy team and by extension  the entirety of U.S.\ foreign policy is not that much different from Trump’s, he does not understand the fundamental convention of Washington — in the United States the money is made first then the political career (i.e. Bush, Kennedy, Reagan, etc.), in most other nation-state the political career precedes the accruement of wealth (i.e. Lukashenko, Putin, Yanukovych).
\end{justify}

\end{document}

