%%%%%%%%%%%%%%%%%%%%%%%%%%%%%%%%%%%%%%%%%%%%%%%%%%%%%%%%%%%%%%%%%%%%%%%%%%%%%%%%%%%%%%%%%%%%%%%%%%%%%%%%%%%%%%%%%%%%%%%%%%%%%%%%%%%%%%%%%%%%%%%%%%%%%%%%%%%%%%%%%%%%%%%%%%%%%%%%%%%%%%%%%%%%
% Written By Michael Brodskiy
% Class: International Relations
% Professor: J. Kropf
%%%%%%%%%%%%%%%%%%%%%%%%%%%%%%%%%%%%%%%%%%%%%%%%%%%%%%%%%%%%%%%%%%%%%%%%%%%%%%%%%%%%%%%%%%%%%%%%%%%%%%%%%%%%%%%%%%%%%%%%%%%%%%%%%%%%%%%%%%%%%%%%%%%%%%%%%%%%%%%%%%%%%%%%%%%%%%%%%%%%%%%%%%%%

\documentclass[12pt]{article} 
\usepackage{alphalph}
\usepackage[utf8]{inputenc}
\usepackage[russian,english]{babel}
\usepackage{titling}
\usepackage{amsmath}
\usepackage{graphicx}
\usepackage{enumitem}
\usepackage{amssymb}
\usepackage[super]{nth}
\usepackage{everysel}
\usepackage{ragged2e}
\usepackage{geometry}
\usepackage{fancyhdr}
\usepackage{cancel}
\usepackage{siunitx}
\usepackage{xcolor}
\usepackage{setspace}
\doublespacing
\geometry{top=1.0in,bottom=1.0in,left=1.0in,right=1.0in}
\newcommand{\subtitle}[1]{%
  \posttitle{%
    \par\end{center}
    \begin{center}\large#1\end{center}
    \vskip0.5em}%

}
\usepackage{hyperref}
\hypersetup{
colorlinks=true,
linkcolor=blue,
filecolor=magenta,      
urlcolor=blue,
citecolor=blue,
}

\urlstyle{same}


\title{Journal One}
\date{June 15, 2021}
\author{Michael Brodskiy\\ \small Instructor: Prof. Kropf}

% Mathematical Operations:

% Sum: $$\sum_{n=a}^{b} f(x) $$
% Integral: $$\int_{lower}^{upper} f(x) dx$$
% Limit: $$\lim_{x\to\infty} f(x)$$

\begin{document}

\flushleft Michael Brodskiy

\begin{center}

 \underline{Journal One}

\end{center}

\begin{justify}

  \paragraph{} Stephen Walt's “\textbf{5 Big Questions}” are:
  \singlespacing
  \begin{enumerate}
    \item \textbf{Where is the EU project headed?}
      \begin{itemize}
        \item Historically Europe is known as the continent predominantly associated with social experimentation. Europe, as the provenance of Bakunin's \selectlanguage{russian}\emph{Государственность и Анархия}, Gentile's \emph{Idee Fondamentali}, and Marx's \emph{Das Kapital}, became the \nth{20} century polygon of communism and fascism. The EU project is the manifestation of common interests involving nation-state without shared commonality and is currently the subject of divisive scrutiny as language and cultural barriers coupled with shared responsibilities of discrepant economic institutes enable outsider active measures to be more effective than in previous trans-nation state unions. Likewise, “\emph{doubts have arisen about} \dots [the] \emph{long-term future} [of the EU]”. 
      \end{itemize}
    \item \textbf{If China’s power continues to rise, how easy will it be to get Asian states to balance against it?}
      \begin{itemize}
        \item With China positioning itself as the superpower of the Asian nation states the formation of military and intelligence alliances will balance the correlation of forces across the First Island Chains against the mainland in accordance with the theories of tellulocracies vs thalassocracies. The struggle to contain Chinese aggression will add stress on US relations with partners like India, Japan, Korea, and the Phillippines just as it did with European partners during the infancy of, and contemporary, NATO.
      \end{itemize}
    \item \textbf{What’s the relationship between U.S. defense spending, the deficit, and America’s economic health and well-being?}
      \begin{itemize}
        \item In the post 9/11 world, the United States should not be categorized as a nation state but rather a national-security state in which the military industry complex along with intelligence producers and consumers are the most recognizable recession-proof (\emph{Walt's article is dated July 2010}) businesses.
      \end{itemize}
    \item \textbf{If the U.S. disengaged from key areas in the Muslim world — most notably Iraq and Afghanistan — would the threat of anti-American terrorism rise or fall?}
  \begin{itemize}
    \item Anti-American terrorism will exist as long as the United States maintains unconditional support for Israel. (\emph{al Qaeda still posed a significant national security threat and UBL was not yet liquidated when this article was published}). United States has begun a withdrawal from Afghanistan and although ANA will most likely be able to maintain a secure Bagram airfield and Kabul, remote tribal areas like Northern Kunduz and  Panshjir Valley will become Talib inundated like pre-2001.  
      \end{itemize}
    \item \textbf{Is the era of U.S. primacy over? How will the end of post-Cold War primacy affect its grand strategy and foreign policy?}
  \begin{itemize}
    \item The void left by the disintegration of the Soviet Union has not yet been assumed by China or Russia, nevertheless “\emph{[t]o succeed U.S. diplomacy \dots will have to be more nuanced, attentive, and flexible than it was in the earlier era of clear U.S. dominance}” this will require extreme professionalism in an increase of intelligence operations in an ever changing field to provide senior US policymakers with a complete perspective of all geopolitical environments.
      \end{itemize}
  \end{enumerate}
  \doublespacing
  \vspace{20pt}
  \hline
  \paragraph{} The most important question to consider would be the role of the U.S. as a global superpower for the foreseeable and long-term future. This question encompasses the course of action U.S. foreign policy will assume with respect to the initiated withdrawal from Afghanistan, suppression of Chinese aggression/counter-aggression, and partnership with the EU — the fourth, second, and first questions, respectively. A domestic contemporary realist \textbf{system-level analysis} suggests the inevitability of the importance of utilizing up-to-date tradecraft to supply senior U.S. policymakers with the timely intelligence of the utmost professional quality. In lockstep with this global paradigm shift of open source and clandestine intelligence gathering \href{https://www.newsweek.com/exclusive-inside-militarys-secret-undercover-army-1591881}{new} methods and means must be exploited to secure U.S. hegemony. While the People's Republic of China and the Russian Federation lack a military industry complex of world conquest proportions, they exploit non-state actors and private military contractors to undermine the weakest links of U.S. alliances such as the disenfranchisement of, or even marginalization among, NATO and/or EU nation states. Sanctioning does little, as history shows, to stimulate a constructive dialogue between heads of state or their subordinate representatives. Diplomatic relations of an international nature cannot be achieved without dependence on solid intelligence work. “\emph{If you know the enemy and know yourself, you need not fear the result of a hundred battles. If you know yourself but not the enemy, for every victory gained you will also suffer a defeat. If you know neither the enemy nor yourself, you will succumb in every battle}.” ― Sun Tzu, “\emph{The Art of War}”. Perhaps an answer to the question of U.S. primacy should be to continue to strengthen objective realism in diplomatic discourse because the unilateral political ideology of the U.S.'s most probable geopolitical adversaries prohibits deviationism and therefore limit their capabilities on domestic and international stages.  
\end{justify}

\end{document}

