%%%%%%%%%%%%%%%%%%%%%%%%%%%%%%%%%%%%%%%%%%%%%%%%%%%%%%%%%%%%%%%%%%%%%%%%%%%%%%%%%%%%%%%%%%%%%%%%%%%%%%%%%%%%%%%%%%%%%%%%%%%%%%%%%%%%%%%%%%%%%%%%%%%%%%%%%%%%%%%%%%%%%%%%%%%%%%%%%%%%%%%%%%%%
% Written By Michael Brodskiy
% Class: International Relations
% Professor: J. Kropf
%%%%%%%%%%%%%%%%%%%%%%%%%%%%%%%%%%%%%%%%%%%%%%%%%%%%%%%%%%%%%%%%%%%%%%%%%%%%%%%%%%%%%%%%%%%%%%%%%%%%%%%%%%%%%%%%%%%%%%%%%%%%%%%%%%%%%%%%%%%%%%%%%%%%%%%%%%%%%%%%%%%%%%%%%%%%%%%%%%%%%%%%%%%%

\documentclass[12pt]{article} 
\usepackage{alphalph}
\usepackage[utf8]{inputenc}
\usepackage[russian,english]{babel}
\usepackage{titling}
\usepackage{amsmath}
\usepackage{graphicx}
\usepackage{enumitem}
\usepackage{amssymb}
\usepackage[super]{nth}
\usepackage{everysel}
\usepackage{ragged2e}
\usepackage{geometry}
\usepackage{fancyhdr}
\usepackage{cancel}
\usepackage{siunitx}
\usepackage{xcolor}
\usepackage{setspace}
\doublespacing
\geometry{top=1.0in,bottom=1.0in,left=1.0in,right=1.0in}
\newcommand{\subtitle}[1]{%
  \posttitle{%
    \par\end{center}
    \begin{center}\large#1\end{center}
    \vskip0.5em}%

}
\usepackage{hyperref}
\hypersetup{
colorlinks=true,
linkcolor=blue,
filecolor=magenta,      
urlcolor=blue,
citecolor=blue,
}

\urlstyle{same}


\title{Journal One}
\date{June 15, 2021}
\author{Michael Brodskiy\\ \small Instructor: Prof. Kropf}

% Mathematical Operations:

% Sum: $$\sum_{n=a}^{b} f(x) $$
% Integral: $$\int_{lower}^{upper} f(x) dx$$
% Limit: $$\lim_{x\to\infty} f(x)$$

\begin{document}

\flushleft Michael Brodskiy

\begin{center}

 \underline{Journal Five}

\end{center}

\begin{justify}
  \paragraph{} In his 1996 book \emph{Demonic Males}: \emph{Apes and the Origins of Human Violence}, Harvard anthropologist Richard Wrangham asserts the idea of the demonic male under the pretense that evolution of homo sapiens is the underlying factor behind the historical perpetuation of war. While this idea is not particularly revolutionary, Wrangham's work inspired the coining of the term “\emph{chimpicide}” by a fellow Vardian psychologist; a term that encapsulates the idea of chimpanzee tribalism (i.e. \emph{intercommunity killings}) as seen by researchers in Uganda. The compromise of the chimpanzee's territorial integrity at the hands of poachers, farmers, and other humans in tandem with environmental and even cultural factors instigate chimpanzee violence. Many parallels can be drawn between the sources and nature of chimpanzee and human violence however, according to John Horgan, the stochastic variables measuring chimpanzee coalitionary killings are skewed in their representation and therefore often misinterpreted. Yet, coalitionary killings are only potentially rare — a lack of evidence is an evidence of nothing. Horgan also highlights the relatively less aggressive nature of the pygmy chimps and an evolutionary significant piece of information concerning the first hominids and their relatively less aggressive nature with respect to both pygmy chimps. Although Horgan brings up many arguments worth considering, it is difficult to concede that the demonic male theory is fallacious even if it appears easy to debunk. Contrary to Horgan's skepticism, consider a bar — a wholesome place where homo sapiens gather to enjoy themselves outside the confines of work, school, or otherwise societal obligations nonetheless, a bar can often have outbursts of violence. In this manner, if an environment where people congregate to relax among other people seeking the same result can turn violent then the demonic male theory must hold true as to the inherent violent nature of hominids. Chimpanzees and homo sapiens alike, when threatened by territorial considerations, limited resources, or otherwise self-preservation will always turn into a demonic male using violence as a means to an end. 
\end{justify}

\end{document}

