%%%%%%%%%%%%%%%%%%%%%%%%%%%%%%%%%%%%%%%%%%%%%%%%%%%%%%%%%%%%%%%%%%%%%%%%%%%%%%%%%%%%%%%%%%%%%%%%%%%%%%%%%%%%%%%%%%%%%%%%%%%%%%%%%%%%%%%%%%%%%%%%%%%%%%%%%%%%%%%%%%%%%%%%%%%%%%%%%%%%%%%%%%%%
% Written By Michael Brodskiy
% Class: International Relations
% Professor: J. Kropf
%%%%%%%%%%%%%%%%%%%%%%%%%%%%%%%%%%%%%%%%%%%%%%%%%%%%%%%%%%%%%%%%%%%%%%%%%%%%%%%%%%%%%%%%%%%%%%%%%%%%%%%%%%%%%%%%%%%%%%%%%%%%%%%%%%%%%%%%%%%%%%%%%%%%%%%%%%%%%%%%%%%%%%%%%%%%%%%%%%%%%%%%%%%%

\documentclass[12pt]{article} 
\usepackage{alphalph}
\usepackage[utf8]{inputenc}
\usepackage[russian,english]{babel}
\usepackage{titling}
\usepackage{amsmath}
\usepackage{graphicx}
\usepackage{enumitem}
\usepackage{amssymb}
\usepackage[super]{nth}
\usepackage{everysel}
\usepackage{ragged2e}
\usepackage{geometry}
\usepackage{fancyhdr}
\usepackage{cancel}
\usepackage{siunitx}
\usepackage{xcolor}
\usepackage{setspace}
\doublespacing
\geometry{top=1.0in,bottom=1.0in,left=1.0in,right=1.0in}
\newcommand{\subtitle}[1]{%
  \posttitle{%
    \par\end{center}
    \begin{center}\large#1\end{center}
    \vskip0.5em}%

}
\usepackage{hyperref}
\hypersetup{
colorlinks=true,
linkcolor=blue,
filecolor=magenta,      
urlcolor=blue,
citecolor=blue,
}

\urlstyle{same}

% Mathematical Operations:

% Sum: $$\sum_{n=a}^{b} f(x) $$
% Integral: $$\int_{lower}^{upper} f(x) dx$$
% Limit: $$\lim_{x\to\infty} f(x)$$

\begin{document}

\flushleft Michael Brodskiy

\begin{center}

 \underline{Journal Nine}

\end{center}

\begin{justify}
  \paragraph{} After reading and considering the arguments of Aleklett, Cavaney, Flavin, Kaufman, and Smil's articles in World Watch's “Peak Oil Forum”, it becomes apparent that Kjell Aleklett's premises are arguably most inadequate with our course textbook's definition of the Security Dilemma because of his purely scientific perspective, considering his tenure as a Professor of Physics. Aleklett, in the spirit of a technocrat, asks the proper questions and seeks to construct a sound conclusion from a factual basis. According to Aleklett, “[o]nly a few countries—Saudi Arabia, Iraq, Kuwait, United Arab Emirates, Kazakhstan, and Bolivia—have the potential to produce more oil than before”, however an obvious candidate deserving to be added to this list is the Russian Federation due to its largely untapped \emph{potential} oil reserves in the thawing Siberian taiga. Aleklett's main argument isn't against oil as means of commercial sustainment but rather as a predominant commodity. While many governments and NGOs alike would concede that a substitute for oil is in order, no government would risk eschewing oil for fear of the military compromises that decision would entail. If not for Aleklett's multiple degrees of separation from the United States government, perhaps his argument wouldn't be as credible; the remainder of the articles were produced by representatives of American Petroleum Institute, WorldWatch Institute, and United States Government Consultancies. More recently, considering the trajectory of U.S. foreign policy surrounding the removal of sanctions against the Nord Stream pipeline, oil's value as an economic, as well as political metric, becomes apparent in a contemporary geopolitical context. 
  \indent\paragraph{} Another interesting quote deserving of mention was in the editor, Tom Prugh's, prologue, “[w]hile no one can know what [\emph{year}] will be like, it's worth remembering that some people find the prospect of civilizational collapse deliciously fascinating”, regarding the human condition as it relates to impending doom — this perhaps is the underlining force driving Security Theory and how its application affects International Relations.
\end{justify}

\end{document}

