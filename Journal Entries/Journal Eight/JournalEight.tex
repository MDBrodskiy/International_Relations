%%%%%%%%%%%%%%%%%%%%%%%%%%%%%%%%%%%%%%%%%%%%%%%%%%%%%%%%%%%%%%%%%%%%%%%%%%%%%%%%%%%%%%%%%%%%%%%%%%%%%%%%%%%%%%%%%%%%%%%%%%%%%%%%%%%%%%%%%%%%%%%%%%%%%%%%%%%%%%%%%%%%%%%%%%%%%%%%%%%%%%%%%%%%
% Written By Michael Brodskiy
% Class: International Relations
% Professor: J. Kropf
%%%%%%%%%%%%%%%%%%%%%%%%%%%%%%%%%%%%%%%%%%%%%%%%%%%%%%%%%%%%%%%%%%%%%%%%%%%%%%%%%%%%%%%%%%%%%%%%%%%%%%%%%%%%%%%%%%%%%%%%%%%%%%%%%%%%%%%%%%%%%%%%%%%%%%%%%%%%%%%%%%%%%%%%%%%%%%%%%%%%%%%%%%%%

\documentclass[12pt]{article} 
\usepackage{alphalph}
\usepackage[utf8]{inputenc}
\usepackage[russian,english]{babel}
\usepackage{titling}
\usepackage{amsmath}
\usepackage{graphicx}
\usepackage{enumitem}
\usepackage{amssymb}
\usepackage[super]{nth}
\usepackage{everysel}
\usepackage{ragged2e}
\usepackage{geometry}
\usepackage{fancyhdr}
\usepackage{cancel}
\usepackage{siunitx}
\usepackage{xcolor}
\usepackage{setspace}
\doublespacing
\geometry{top=1.0in,bottom=1.0in,left=1.0in,right=1.0in}
\newcommand{\subtitle}[1]{%
  \posttitle{%
    \par\end{center}
    \begin{center}\large#1\end{center}
    \vskip0.5em}%

}
\usepackage{hyperref}
\hypersetup{
colorlinks=true,
linkcolor=blue,
filecolor=magenta,      
urlcolor=blue,
citecolor=blue,
}

\urlstyle{same}


\title{Journal One}
\date{June 15, 2021}
\author{Michael Brodskiy\\ \small Instructor: Prof. Kropf}

% Mathematical Operations:

% Sum: $$\sum_{n=a}^{b} f(x) $$
% Integral: $$\int_{lower}^{upper} f(x) dx$$
% Limit: $$\lim_{x\to\infty} f(x)$$

\begin{document}

\flushleft Michael Brodskiy

\begin{center}

 \underline{Journal Eight}

\end{center}

\begin{justify}
  \paragraph{} Of the three types of “\emph{insecurity}” outlined by Helena Norberg-Hodge, job insecurity most probably challenges the global status quo. The reason as to why job insecurity trumps political and psychological insecurity is because of the need for people to survive, in many cases at any subsistence level whatsoever. Before an individual can assess and confirm their political identity or tend to their psychological or even mental health they must have a roof over their head and consistent access to warm nutritionally sustainable food. As a result of the ubiquity of proponents of the economies of scale “local econom[ies] tend to experience a net loss of jobs, as smaller competitors that tend to be more dependent on human labour go out of business” when “businesses are centralised and scaled up”. In this manner it becomes apparent that the permeation of globalization, while strengthening international imports and exports, “undermine the livelihoods” of the proletariat. Also, as Norberg-Hodge suggests, the “\emph{fourth branch}” of the government, the media, can utilize the fear associated with job loss to achieve a certain agenda. In essence, the media's, or any organized operator's, ability to manipulate the public by implying job instabilities highlight the ascendance of job insecurity above both political and psychological. To quote Joseph Stalin author of \emph{Historical and Dialectical Materialism}, “[i]t is difficult for me to imagine what “\emph{personal liberty}” is enjoyed by an unemployed person, who goes about hungry, and cannot find employment \dots [r]eal liberty can exist only where there is no unemployment and poverty, where a man is not haunted by the fear of being tomorrow deprived of work, of home and of bread”. Obviously this thought predates the contemporary prominence of globalization, but it nonetheless indicates the importance of job security. Consider the higher crime rates associated with high unemployment rates in American inner urban regions. When people have work they have a societal role — a mutual obligation that provides compensation for exploitation. Therefore, the biggest challenge to the status quo is job insecurity.
\end{justify}

\end{document}

