%%%%%%%%%%%%%%%%%%%%%%%%%%%%%%%%%%%%%%%%%%%%%%%%%%%%%%%%%%%%%%%%%%%%%%%%%%%%%%%%%%%%%%%%%%%%%%%%%%%%%%%%%%%%%%%%%%%%%%%%%%%%%%%%%%%%%%%%%%%%%%%%%%%%%%%%%%%%%%%%%%%%%%%%%%%%%%%%%%%%%%%%%%%%
% Written By Michael Brodskiy
% Class: International Relations
% Professor: J. Kropf
%%%%%%%%%%%%%%%%%%%%%%%%%%%%%%%%%%%%%%%%%%%%%%%%%%%%%%%%%%%%%%%%%%%%%%%%%%%%%%%%%%%%%%%%%%%%%%%%%%%%%%%%%%%%%%%%%%%%%%%%%%%%%%%%%%%%%%%%%%%%%%%%%%%%%%%%%%%%%%%%%%%%%%%%%%%%%%%%%%%%%%%%%%%%

\documentclass[12pt]{article} 
\usepackage{alphalph}
\usepackage[utf8]{inputenc}
\usepackage[russian,english]{babel}
\usepackage{titling}
\usepackage{amsmath}
\usepackage{graphicx}
\usepackage{enumitem}
\usepackage{amssymb}
\usepackage[super]{nth}
\usepackage{everysel}
\usepackage{ragged2e}
\usepackage{geometry}
\usepackage{fancyhdr}
\usepackage{cancel}
\usepackage{siunitx}
\usepackage{xcolor}
\usepackage{setspace}
\doublespacing
\geometry{top=1.0in,bottom=1.0in,left=1.0in,right=1.0in}
\newcommand{\subtitle}[1]{%
  \posttitle{%
    \par\end{center}
    \begin{center}\large#1\end{center}
    \vskip0.5em}%

}
\usepackage{hyperref}
\hypersetup{
colorlinks=true,
linkcolor=blue,
filecolor=magenta,      
urlcolor=blue,
citecolor=blue,
}

\urlstyle{same}


\title{Journal One}
\date{June 15, 2021}
\author{Michael Brodskiy\\ \small Instructor: Prof. Kropf}

% Mathematical Operations:

% Sum: $$\sum_{n=a}^{b} f(x) $$
% Integral: $$\int_{lower}^{upper} f(x) dx$$
% Limit: $$\lim_{x\to\infty} f(x)$$

\begin{document}

\flushleft Michael Brodskiy

\begin{center}

\underline{Journal Three}

\end{center}

\begin{justify}

  \paragraph{} In Marysia Zalewski's article entitled \emph{Feminist Approaches To International Relations Theory In The Post-Cold War Period} a strong parallel is drawn between `feminist intent' of “\emph{\underline{the Lady}}\footnote{For the remainder of this text Aung San Suu Kyi will the referred to as \emph{the Lady} in the interests of brevity}” Aung San Suu Kyi with respect to international relations as a globally significant woman for her role in providing a voice for the downtrodden, voiceless casualties during the extrajudicial killings of genocides in Burma. Interestingly enough, Zalewski's article fails to mention the Myanmar constitutional crisis and subsequent \href{https://en.wikipedia.org/wiki/2021\_Myanmar\_coup\_d\%27\%C\3\%A9tat}{coup d'\'etat} against \emph{the Lady}'s government earlier this year. The predominant domestic political positions assumed by \emph{the Lady} since her entering the world stage of international relations (the transition of Myanmar from a military stronghold to, loosely, a democracy) was the cause for her arrest following the charges brought against her. Her untimely hiatus from the cabinet of Minister of Foreign Affairs at the hands of the so called “\emph{malestream}” \underline{Burmese paramilitary statesmen}\footnote{The Republic of China is the most probable geopolitical adversary of the United States in southeast Asia and the \href{https://intelnews.org/2021/02/04/01-2949/}{role} of the Communist Party of China in the Myanmar coup d'\'etat in should not be discounted.} should not subtract from her commitments to unequivocally bring democracy to Myanmar. As a globally significant woman, \emph{the Lady} boldly oriented herself as a major proponent of female inclusion in broader geopolitical thought in addition to her underlying political agenda even as early as 1988, before the disintegration of the Soviet Union. \emph{The Lady} is, at least in western culture, embraced as a Ghandi-like figure of womanhood who impacted her own domestic State Assembly, the transnational UN and similar male dominated political institutions. Perhaps her most daring accomplishment was assuming the chair of an office she was responsible for creating such as the National League for Democracy. Her achievements are inarguable and as Zalewski suggests, “peace is not [an] issue here \dots think about her iconic status in relation to liberal feminist visions of feminist inclusion, especially as global political leaders”.

\end{justify}

\end{document}

