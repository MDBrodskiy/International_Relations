%%%%%%%%%%%%%%%%%%%%%%%%%%%%%%%%%%%%%%%%%%%%%%%%%%%%%%%%%%%%%%%%%%%%%%%%%%%%%%%%%%%%%%%%%%%%%%%%%%%%%%%%%%%%%%%%%%%%%%%%%%%%%%%%%%%%%%%%%%%%%%%%%%%%%%%%%%%%%%%%%%%%%%%%%%%%%%%%%%%%%%%%%%%%
% Written By Michael Brodskiy
% Class: International Relations
% Professor: J. Kropf
%%%%%%%%%%%%%%%%%%%%%%%%%%%%%%%%%%%%%%%%%%%%%%%%%%%%%%%%%%%%%%%%%%%%%%%%%%%%%%%%%%%%%%%%%%%%%%%%%%%%%%%%%%%%%%%%%%%%%%%%%%%%%%%%%%%%%%%%%%%%%%%%%%%%%%%%%%%%%%%%%%%%%%%%%%%%%%%%%%%%%%%%%%%%

\documentclass[12pt]{article} 
\usepackage{alphalph}
\usepackage[utf8]{inputenc}
\usepackage[russian,english]{babel}
\usepackage{titling}
\usepackage{amsmath}
\usepackage{graphicx}
\usepackage{enumitem}
\usepackage{amssymb}
\usepackage[super]{nth}
\usepackage{everysel}
\usepackage{ragged2e}
\usepackage{geometry}
\usepackage{fancyhdr}
\usepackage{cancel}
\usepackage{siunitx}
\usepackage{xcolor}
\usepackage{setspace}
\doublespacing
\geometry{top=1.0in,bottom=1.0in,left=1.0in,right=1.0in}
\newcommand{\subtitle}[1]{%
  \posttitle{%
    \par\end{center}
    \begin{center}\large#1\end{center}
    \vskip0.5em}%

}
\usepackage{hyperref}
\hypersetup{
colorlinks=true,
linkcolor=blue,
filecolor=magenta,      
urlcolor=blue,
citecolor=blue,
}

\urlstyle{same}


\title{Journal One}
\date{June 15, 2021}
\author{Michael Brodskiy\\ \small Instructor: Prof. Kropf}

% Mathematical Operations:

% Sum: $$\sum_{n=a}^{b} f(x) $$
% Integral: $$\int_{lower}^{upper} f(x) dx$$
% Limit: $$\lim_{x\to\infty} f(x)$$

\begin{document}

\flushleft Michael Brodskiy

\begin{center}

\underline{Journal Four}

\end{center}

\begin{justify}

  \paragraph{} In his essay, \emph{The Clash of Civilizations}, Samuel Huntington outlines a pattern of conflict with respect to shared commonality stating that “the source of conflict in [the] new world will not be primarily ideological or primarily economic \dots the dominating source of conflict will be cultural". This thesis is efficaciously applicable to Afghanistan's role during the September \nth{11} terrorist attacks, more specifically, the ethno-nationalist Pasthun movements of southern Afghanistan following the Soviet withdrawal of economic, military, and political support. Of course the aforementioned ethno-nationalist movements are in euphemistic reference to the formation of the Taliban as a self-dependent, fundamentalist, ethnically unifying cultural group regimented under the tutelage of Mullah Omar and in relative autarchy from the multilateral power struggle in Kabul. Huntington's thesis is sustained by fact that the Pashtun unification in the form of Talib hostility toward Tajiks, Uzbeks, and Turkmens to first control Kandahar, and eventually the entirety of the Afghan Emirate was foremost a cultural source of conflict as opposed to an ideological one. The role of the Taliban providing shelter and protection to UBL's al-Qaeda, and UBL himself, is what warranted the 2001 United States invasion of Afghanistan and unconditional support of the Northern Alliance's Bonapartist coup d'\'etat against the de facto Taliban government; essentially transforming a local cultural conflict in a major global one of the western—non-western orientation. From Huntington, “[i]n the former Soviet Union communists can become democrats \dots but Russians cannot become Estonians”, in this manner since the terrorist attacks of September \nth{11}, the polarization of \emph{incompatible} civilizations such as the ongoing Russian—Ukrainian or western—non-western conflicts, such as the aforementioned invasion of Afghanistan illustrate the complexity of modern realism. Contemporary civilizations most likely to “\emph{clash}” are the former Soviet republics which, while being enticed by western liberal democracy, are unable to overcome cultural conflicts (i.e. Russia—Chechnya, Georgia, Ukraine, etc.).

\end{justify}

\end{document}

