%%%%%%%%%%%%%%%%%%%%%%%%%%%%%%%%%%%%%%%%%%%%%%%%%%%%%%%%%%%%%%%%%%%%%%%%%%%%%%%%%%%%%%%%%%%%%%%%%%%%%%%%%%%%%%%%%%%%%%%%%%%%%%%%%%%%%%%%%%%%%%%%%%%%%%%%%%%%%%%%%%%%%%%%%%%%%%%%%%%%%%%%%%%%
% Written By Michael Brodskiy
% Class: International Relations
% Professor: J. Kropf
%%%%%%%%%%%%%%%%%%%%%%%%%%%%%%%%%%%%%%%%%%%%%%%%%%%%%%%%%%%%%%%%%%%%%%%%%%%%%%%%%%%%%%%%%%%%%%%%%%%%%%%%%%%%%%%%%%%%%%%%%%%%%%%%%%%%%%%%%%%%%%%%%%%%%%%%%%%%%%%%%%%%%%%%%%%%%%%%%%%%%%%%%%%%

\documentclass[12pt]{article} 
\usepackage{alphalph}
\usepackage[utf8]{inputenc}
\usepackage[russian,english]{babel}
\usepackage{titling}
\usepackage{amsmath}
\usepackage{graphicx}
\usepackage{enumitem}
\usepackage{amssymb}
\usepackage[super]{nth}
\usepackage{everysel}
\usepackage{ragged2e}
\usepackage{geometry}
\usepackage{fancyhdr}
\usepackage{cancel}
\usepackage{siunitx}
\usepackage{xcolor}
\usepackage{setspace}
\doublespacing
\geometry{top=1.0in,bottom=1.0in,left=1.0in,right=1.0in}
\newcommand{\subtitle}[1]{%
  \posttitle{%
    \par\end{center}
    \begin{center}\large#1\end{center}
    \vskip0.5em}%

}
\usepackage{hyperref}
\hypersetup{
colorlinks=true,
linkcolor=blue,
filecolor=magenta,      
urlcolor=blue,
citecolor=blue,
}

\urlstyle{same}


\title{Journal Two}
\date{June 17, 2021}
\author{Michael Brodskiy\\ \small Instructor: Prof. Kropf}

% Mathematical Operations:

% Sum: $$\sum_{n=a}^{b} f(x) $$
% Integral: $$\int_{lower}^{upper} f(x) dx$$
% Limit: $$\lim_{x\to\infty} f(x)$$

\begin{document}

\flushleft Michael Brodskiy

\begin{center}

 \underline{Journal Two}

\end{center}

\begin{justify}

  \paragraph{} Throughout his piece, Michael Klare discusses the impending energy crisis, all the while comparing it to the Thirty Year's War and the resulting Treaty of Westphalia. Klare's overarching argument revolves around the notion that the energy landscape of the next 30 (now 20, as the article was published in 2011) years will have a similar outcome as that of the Thirty Year's War — the formation of interdependent nation-states. Though it may seem this is a grim possibility, Klare's analysis of various energy sources is hypercritical, as he did not account for innovation. For example, contemporary technology allows for the renewal of nuclear waste from reactors, essentially making it reusable. In this manner, it seems that Klare's predictions are unlikely, as energy solutions evolve. Additionally, the Thirty Year's War and Klare's thesis are quite dissimilar — except for both lasting 30 years and resulting in the formation of nation-states. Most importantly, the conclusion of the Thirty Year's War, which was fought over a lust for freedom of religion, demarcated positive change, as religious freedom became more widespread, with people being able to migrate to nation-states that suited them; whereas, with Klare's thesis, only regression would occur, as societies would crumble and withdraw. As such, it appears that Klare's idea of the Thirty Year's War being related to the energy crisis are only loosely related.\\
  \indent Given Klare's description, it seems a more appropriate historical example is that of the Cold War arms race. Essentially, the two parallel each other in that geopolitical adversaries were dedicating massive resources to out-develop each other. Similarly, Klare's argument follows the logic that, as environmental conditions deteriorate, competing superpowers (specifically China and the U.S.) will try to defeat each other through more energy-efficient and clean systems. Additionally, unlike the Thirty Year's War, the arms race actually resulted in great scientific advancement. Ergo, it seems that, rather than comparing the energy crisis to the Thirty Year's War, the Cold War arms race seems to be a more plausible comparison.

\end{justify}

\end{document}

