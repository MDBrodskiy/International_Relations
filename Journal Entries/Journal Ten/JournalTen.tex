%%%%%%%%%%%%%%%%%%%%%%%%%%%%%%%%%%%%%%%%%%%%%%%%%%%%%%%%%%%%%%%%%%%%%%%%%%%%%%%%%%%%%%%%%%%%%%%%%%%%%%%%%%%%%%%%%%%%%%%%%%%%%%%%%%%%%%%%%%%%%%%%%%%%%%%%%%%%%%%%%%%%%%%%%%%%%%%%%%%%%%%%%%%%
% Written By Michael Brodskiy
% Class: International Relations
% Professor: J. Kropf
%%%%%%%%%%%%%%%%%%%%%%%%%%%%%%%%%%%%%%%%%%%%%%%%%%%%%%%%%%%%%%%%%%%%%%%%%%%%%%%%%%%%%%%%%%%%%%%%%%%%%%%%%%%%%%%%%%%%%%%%%%%%%%%%%%%%%%%%%%%%%%%%%%%%%%%%%%%%%%%%%%%%%%%%%%%%%%%%%%%%%%%%%%%%

\documentclass[12pt]{article} 
\usepackage{alphalph}
\usepackage[utf8]{inputenc}
\usepackage[russian,english]{babel}
\usepackage{titling}
\usepackage{amsmath}
\usepackage{graphicx}
\usepackage{enumitem}
\usepackage{amssymb}
\usepackage[super]{nth}
\usepackage{everysel}
\usepackage{ragged2e}
\usepackage{geometry}
\usepackage{fancyhdr}
\usepackage{cancel}
\usepackage{siunitx}
\usepackage{xcolor}
\usepackage{setspace}
\doublespacing
\geometry{top=1.0in,bottom=1.0in,left=1.0in,right=1.0in}
\newcommand{\subtitle}[1]{%
  \posttitle{%
    \par\end{center}
    \begin{center}\large#1\end{center}
    \vskip0.5em}%

}
\usepackage{hyperref}
\hypersetup{
colorlinks=true,
linkcolor=blue,
filecolor=magenta,      
urlcolor=blue,
citecolor=blue,
}

\urlstyle{same}


\title{Journal One}
\date{June 15, 2021}
\author{Michael Brodskiy\\ \small Instructor: Prof. Kropf}

% Mathematical Operations:

% Sum: $$\sum_{n=a}^{b} f(x) $$
% Integral: $$\int_{lower}^{upper} f(x) dx$$
% Limit: $$\lim_{x\to\infty} f(x)$$

\begin{document}

\flushleft Michael Brodskiy

\begin{center}

 \underline{Journal Ten}

\end{center}

\begin{justify}
  \paragraph{} Assuming Saad Hafiz is correct, in order to stimulate a culture of tolerance and understanding that will be instrumental in shaping a peaceful \nth{21} century between the West and Muslim world an external threat, one that presents a constant pressure, must be introduced. Consider the geopolitical context of the United States intelligence community's ties with the Muslim World, predominantly Pakistan's ISI. The Soviet sponsored counter-coup, aftermath of Operation Storm-333, and the eventual invasion led to not only the acceptance but overall pity and unconditional support by westerners toward the plight of the Tajiks and native Afghan Pashtuns as the principally oppressed ethnic groups. Outlined by the expression, “the enemy of my enemy is my friend”, western governments couldn't afford not to develop international relations with the Muslim World's jihad against the \underline{shuravi}\footnote{The Persian designation for Soviet. \href{https://en.wikipedia.org/wiki/Shuravi}{Link}.} during the dichotomous cold war. According to the Hindustan Times, \href{https://www.hindustantimes.com/world-news/china-asks-taliban-to-make-clean-break-from-terrorists-return-to-mainstream-101626254774269.html}{Sino-Afghan relations} are back channeled through the Taliban as opposed to the democratic Afghan national government. “[T]he Taliban said that the group would guarantee the safety of Chinese investors and workers in Afghanistan and would not host Uyghur militants from the Xinjiang province”, however, Chinese domestic oppression of religious minorities are akin to the Soviet oppression of religious minorities, if not outright worse, so if there was ever a unifying call for a third global jihad in the last half century this would be it. Considering the first global jihad was against Marxism-Leninism (i.e. Soviet expansion, communist solidarity), the second against the West (ubiquitous influence, imperial hubris, etc.), and the third could potentially be against the East for an amalgamation of the previous two causes. If relations between the West and Muslim World were to return to their cold war era of normalcy, it would most probably be as a result of China replacing the United States as the chiefest nation-state actor in Afghanistan. Only a new geopolitical rival can rekindle old relationships.
\end{justify}

\end{document}

