%%%%%%%%%%%%%%%%%%%%%%%%%%%%%%%%%%%%%%%%%%%%%%%%%%%%%%%%%%%%%%%%%%%%%%%%%%%%%%%%%%%%%%%%%%%%%%%%%%%%%%%%%%%%%%%%%%%%%%%%%%%%%%%%%%%%%%%%%%%%%%%%%%%%%%%%%%%%%%%%%%%%%%%%%%%%%%%%%%%%%%%%%%%%
% Written By Michael Brodskiy
% Class: International Relations
% Professor: J. Kropf
%%%%%%%%%%%%%%%%%%%%%%%%%%%%%%%%%%%%%%%%%%%%%%%%%%%%%%%%%%%%%%%%%%%%%%%%%%%%%%%%%%%%%%%%%%%%%%%%%%%%%%%%%%%%%%%%%%%%%%%%%%%%%%%%%%%%%%%%%%%%%%%%%%%%%%%%%%%%%%%%%%%%%%%%%%%%%%%%%%%%%%%%%%%%

\documentclass[12pt]{article} 
\usepackage{alphalph}
\usepackage[utf8]{inputenc}
\usepackage[russian,english]{babel}
\usepackage{titling}
\usepackage{amsmath}
\usepackage{graphicx}
\usepackage{enumitem}
\usepackage{amssymb}
\usepackage[super]{nth}
\usepackage{everysel}
\usepackage{ragged2e}
\usepackage{geometry}
\usepackage{fancyhdr}
\usepackage{cancel}
\usepackage{siunitx}
\usepackage{xcolor}
\usepackage{setspace}
\doublespacing
\geometry{top=1.0in,bottom=1.0in,left=1.0in,right=1.0in}
\newcommand{\subtitle}[1]{%
  \posttitle{%
    \par\end{center}
    \begin{center}\large#1\end{center}
    \vskip0.5em}%

}
\usepackage{hyperref}
\hypersetup{
colorlinks=true,
linkcolor=blue,
filecolor=magenta,      
urlcolor=blue,
citecolor=blue,
}

\urlstyle{same}


\title{Journal One}
\date{June 15, 2021}
\author{Michael Brodskiy\\ \small Instructor: Prof. Kropf}

% Mathematical Operations:

% Sum: $$\sum_{n=a}^{b} f(x) $$
% Integral: $$\int_{lower}^{upper} f(x) dx$$
% Limit: $$\lim_{x\to\infty} f(x)$$

\begin{document}

\flushleft Michael Brodskiy

\begin{center}

 \underline{Journal Six}

\end{center}

\begin{justify}
  \paragraph{} While Ben Dangl's arguments are entertaining, they are merely that, a source of textual entertainment without a realist geopolitical consideration. Looking “\emph{back to the history and political vision of the demand for a} [\emph{U}.\emph{S}. \emph{Department of Peace}] \emph{offers methods for transforming our militaristic culture and provides a political toolbox for peace rather than perpetual war}”. However, looking back two decades to the post 9/11 shift in American domestic and foreign policy (PATRIOT Act, Wolfowitz Doctrine, etc.) it becomes obvious that the United States made the irreversible transition from a liberal democratic nation-state to a liberal democratic national-security state. What's more demoralizing is to consider the main argument of Douglas Valentine's \emph{The Phoenix Program: America's Use of Terror in Vietnam}. In this text Valentine claims that the use of Provincial Reconaissance Units and Field Police to centralize gathered information regarding VCI whereabouts and activities at PICCs (Province Intelligence Coordinating Committees) is analogous to the utilization of fusion centers in the interest of law enforcement organizations. Therefore, based on the domestic and foreign policy realignment toward prioritizing security considerations it should be safe to assume that there never will be a U.S. Department of Peace, or that it, if created, will never be able to provide a check on the power of the War Office as envisioned by Benjamin Rush. Simply put, the genie is already out of the bottle — the privatization and corporate profiting overlooked by the Congress and enabled by the Department of Defense will forever dwarf the scope of Department of Peace advocates outside the scope of Washington's private military company special interest group lobbyists. On the contrary, however, international relations is solely a tool preceding diplomacy, if diplomacy fails then a corresponding response is preventative militaristic intervention, privatized or otherwise, to secure national interests. Therefore, a department such as the Department of Peace could not be created in the U.S.\ government as it would conflict in the global mission of U.S.\ unilateralism.
\end{justify}

\end{document}

