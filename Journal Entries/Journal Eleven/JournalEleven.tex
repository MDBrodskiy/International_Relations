%%%%%%%%%%%%%%%%%%%%%%%%%%%%%%%%%%%%%%%%%%%%%%%%%%%%%%%%%%%%%%%%%%%%%%%%%%%%%%%%%%%%%%%%%%%%%%%%%%%%%%%%%%%%%%%%%%%%%%%%%%%%%%%%%%%%%%%%%%%%%%%%%%%%%%%%%%%%%%%%%%%%%%%%%%%%%%%%%%%%%%%%%%%%
% Written By Michael Brodskiy
% Class: International Relations
% Professor: J. Kropf
%%%%%%%%%%%%%%%%%%%%%%%%%%%%%%%%%%%%%%%%%%%%%%%%%%%%%%%%%%%%%%%%%%%%%%%%%%%%%%%%%%%%%%%%%%%%%%%%%%%%%%%%%%%%%%%%%%%%%%%%%%%%%%%%%%%%%%%%%%%%%%%%%%%%%%%%%%%%%%%%%%%%%%%%%%%%%%%%%%%%%%%%%%%%

\documentclass[12pt]{article} 
\usepackage{alphalph}
\usepackage[utf8]{inputenc}
\usepackage[russian,english]{babel}
\usepackage{titling}
\usepackage{amsmath}
\usepackage{graphicx}
\usepackage{enumitem}
\usepackage{amssymb}
\usepackage[super]{nth}
\usepackage{everysel}
\usepackage{ragged2e}
\usepackage{geometry}
\usepackage{fancyhdr}
\usepackage{cancel}
\usepackage{siunitx}
\usepackage{xcolor}
\usepackage{setspace}
\doublespacing
\geometry{top=1.0in,bottom=1.0in,left=1.0in,right=1.0in}
\newcommand{\subtitle}[1]{%
  \posttitle{%
    \par\end{center}
    \begin{center}\large#1\end{center}
    \vskip0.5em}%

}
\usepackage{hyperref}
\hypersetup{
colorlinks=true,
linkcolor=blue,
filecolor=magenta,      
urlcolor=blue,
citecolor=blue,
}

\urlstyle{same}


\title{Journal One}
\date{June 15, 2021}
\author{Michael Brodskiy\\ \small Instructor: Prof. Kropf}

% Mathematical Operations:

% Sum: $$\sum_{n=a}^{b} f(x) $$
% Integral: $$\int_{lower}^{upper} f(x) dx$$
% Limit: $$\lim_{x\to\infty} f(x)$$

\begin{document}

\flushleft Michael Brodskiy

\begin{center}

 \underline{Journal Eleven}

\end{center}

\begin{justify}
  \paragraph{} Jason W. Moore's response to the interviewing journalist's inquiry regarding the transition from feudalism to early capitalism as it relates to the ecological crisis is essentially two-fold, as it incorporates premises analogizing the mid-fourteenth century's developmentally driven climate shifts and paradoxically unmitigable demise of agricultural yields. The modern United States, and, moreover, the world is experiencing an anomaly of measurable seasonal climate patterns, analogous to the “Little Ice Age”, coupled with the concept of “peak oil” to the backdrop of a plague, not the bubonic plague as was antecedently the case, but a plague of coincidentally correlative oriental origin. Highlighted by the statement, “epidemic disease was not an ``output'' or a ``footprint'' of the feudal system and its crises \dots [but] a civilizational vortex of class struggle on the land, subsistence crises, and much beyond”, the shift to feudalism is often cited by proponents of Marxist-Leninist philosophy and often by Marx himself to demonstrate the history of dialectical materialism — in essence however, the political transition from feudalism was a transition in the distribution of wealth itself. If the agrarian, cultural, and public health crises of the fourteenth century are any indication of the future, then the current state of international affairs suggest a new transition in global political thought; perhaps accompanied by the ubiquity of digitized technologies prevalent both in domestic and commercial environments. A complementary consideration to the aforementioned plagues is the “skyrocketing rates of cancer \dots no one really seems to understand” relative to past generational health data sets (although scarcely documented), it is the belief of many that cancerous endocrines were not as robust as they are now. What's even more interesting is that, if global industrialization is the most probable culprit, the documentation of cancer mortality rates during the later existence of the USSR, the most rapidly industrialized country in history, were “among the highest registered in the world”\footnote{\href{https://pubmed.ncbi.nlm.nih.gov/8262675/}{source}}. 
\end{justify}

\end{document}

