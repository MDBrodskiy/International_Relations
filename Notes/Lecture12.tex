%%%%%%%%%%%%%%%%%%%%%%%%%%%%%%%%%%%%%%%%%%%%%%%%%%%%%%%%%%%%%%%%%%%%%%%%%%%%%%%%%%%%%%%%%%%%%%%%%%%%%%%%%%%%%%%%%%%%%%%%%%%%%%%%%%%%%%%%%%%%%%%%%%%%%%%%%%%%%%%%%%%%%%%%%%%%%%%%%%%%%%%%%%%%
% Written By Michael Brodskiy
% Class: International Relations
% Professor: J. Kropf
%%%%%%%%%%%%%%%%%%%%%%%%%%%%%%%%%%%%%%%%%%%%%%%%%%%%%%%%%%%%%%%%%%%%%%%%%%%%%%%%%%%%%%%%%%%%%%%%%%%%%%%%%%%%%%%%%%%%%%%%%%%%%%%%%%%%%%%%%%%%%%%%%%%%%%%%%%%%%%%%%%%%%%%%%%%%%%%%%%%%%%%%%%%%

\documentclass[12pt]{article} 
\usepackage{alphalph}
\usepackage[utf8]{inputenc}
\usepackage[russian,english]{babel}
\usepackage{titling}
\usepackage{amsmath}
\usepackage{graphicx}
\usepackage{enumitem}
\usepackage{amssymb}
\usepackage[super]{nth}
\usepackage{everysel}
\usepackage{ragged2e}
\usepackage{geometry}
\usepackage{fancyhdr}
\usepackage{cancel}
\usepackage{siunitx}
\usepackage{xcolor}
\geometry{top=1.0in,bottom=1.0in,left=1.0in,right=1.0in}
\newcommand{\subtitle}[1]{%
  \posttitle{%
    \par\end{center}
    \begin{center}\large#1\end{center}
    \vskip0.5em}%

}
\usepackage{hyperref}
\hypersetup{
colorlinks=true,
linkcolor=blue,
filecolor=magenta,      
urlcolor=blue,
citecolor=blue,
}

\urlstyle{same}


\title{Lecture 12 Notes}
\date{July 8, 2021}
\author{Michael Brodskiy\\ \small Instructor: Prof. Kropf}

% Mathematical Operations:

% Sum: $$\sum_{n=a}^{b} f(x) $$
% Integral: $$\int_{lower}^{upper} f(x) dx$$
% Limit: $$\lim_{x\to\infty} f(x)$$

\begin{document}

    \maketitle

    \begin{enumerate}

      \item Breton Woods — The Bretton Woods system of monetary management established the rules for commercial and financial relations among the world's major industrial states in the mid-\nth{20} century

      \item The Great Depression and World War Two led to a collapse of banks and trade

      \item The US emerged from this crisis with half of the globe’s industrial production and four-fifths of its gold reserves. US negotiators were in a good position to ensure the new Bretton Woods rules favored continued US economic dominance

      \item The Bretton Woods vision was to create public institutions to anchor each of the three pillars of global economic activity: 

        \begin{enumerate}

          \item Production — World Bank: Designed to help with reconstruction after the war and to assist long-term production in poorer countries

          \item Finance — International Monetary Fund (IMF): Oversees the international and financial and monetary order (loans)

          \item Trade — GATT, later changed to the World Trade Organization (WTO): The institution set up to free restrictions on trade

        \end{enumerate}

      \item Theoretical Aspects of Neoliberalism

        \begin{enumerate}

          \item Rule of the “free market”

          \item Cuts in social services (SAPs)

          \item De-regulation of businesses and corporations

          \item Privatization of public services

          \item Elimination of the concept of “community” or “common good”

        \end{enumerate}

      \item Globalization — Terms and Concepts

        \begin{enumerate}

          \item Less Developed Countries (LDCs) — Highly disadvantaged and have a limited ability to advance their interests and compete with the economic giants in the North

          \item Newly Industrializing Countries (NICs) — Usually placed by analysts as being in the South, but in some cases can be classified as a developed market economy (i.e. South Korea, Portugal, Mexico)

          \item Least Developed Countries (LLDCs) — The lowest third of the low- income countries and are experiencing a decline in absolute conditions over the last 2 to 3 decades

        \end{enumerate}

      \item Approaches to Development

        \begin{enumerate}

          \item Liberal — Development can be achieved within the existing structure. Focus is on GNP, trade levels, employment and wage statistics. Recipe for success? Free trade, free investment, and other unimpeded economic exchange will eventually create prosperity for all.

          \item Structuralist — Development can only be achieved when the politico-economic organization of the world is radically altered to bring development to the LDCs

            \begin{enumerate}

              \item Dependencia Model

              \item Primary products/raw materials

              \item Neo-Colonialism

            \end{enumerate}

          \item Mercantilist — Political considerations govern international economics. Trade, investment, and aid policies of the North are meant to help the North, not the LDCs

        \end{enumerate}

      \item Globalization has enhanced countries’ needs for capital, which can be used to supplement their own internal efforts to improve socioeconomic conditions. Countries need hard currency such as dollars, euros, and yen, which are acceptable in private channels of international economics

      \item The Four Main Sources of Hard Currency

        \begin{enumerate}

          \item Loans — Usually extended by private or government sources (Problem: leads to debt crises)

            \begin{enumerate}

              \item Economic ramifications of loans: bank failures in the North and inability to pay for social and economic development in the South due to large debts

              \item Political ramifications of loans: strains between lending and borrowing countries and political instability in countries that are struggling to pay off debt

            \end{enumerate}

          \item Private Investment through MNCs (Problem: LDCs are disadvantaged compared to the countries of the North)

            \begin{enumerate}

              \item One study showed that average LDC balance for the period of 1984$-$1990 showed a net inflow of \$132.1 billion in investment capital into LDCs but an outflow of \$97.6 billion in profit taking — for a net capital inflow of \$34.5 billion

              \item However, LLDCs fared poorly. Example? Africa, which had a net investment inflow of \$8.9 billion, but a net profits outflow of \$20.6 billion. Thus MNC activity in Africa cost the continent \$11.7 billion

            \end{enumerate}

          \item Trade — Export earnings do provide hard currency (Problems: a. LDCs command only 28\% of world export market, b. Most LDCs suffer from chronic trade deficits, c. Heavy dependence on other countries for primary products)

          \item Foreign Aid — Limitations include:

            \begin{enumerate}

              \item Political Considerations — Most often given on the basis of political-military interest and not development needs

              \item Military Content — In the 1980s and 90s, approximately half of all U.S. aid involved military transfers.

              \item Measuring Recipient per Capita Aid — In the 90s, LDCs received \$11.80 per capita (Israel received \$617 per capita) 

              \item Donor Aid Relative to Wealth — Of the top 18 industrialized countries, the U.S. was \#1 in total dollars given (11.4 billion in 1990), but ranked \#17 in amount given as total percentage of GNP

              \item The Way Aid is Applied — Often used to fund highly symbolic but economically unwise projects such as airports, sports arenas, or large government buildings. Inefficiency and corruption also a problem
                
            \end{enumerate}

        \end{enumerate}

      \item Alternatives to Economic Globalization

        \begin{enumerate}

          \item New Democracy — An egalitarian approach to economics; every person is a participant in the economy

          \item Subsidiarity — “Localizing” purchases improves the livability of the local economy and also reduces the distance raw material and final products travel to meet end users

          \item Ecological Sustainability — Limit consumption and exploitation.  Conserving resources allows us to meet today’s needs without compromising our ability to meet tomorrows

          \item Common Heritage — Ecological resources (the commons) cannot be monopolized.  History, culture, and civil services belong to everyone

          \item Diversity of Indigenous Peoples — The collective and individual right to maintain and develop our distinct identities and characteristics

          \item Human Rights — “A standard of living adequate for \dots health and well-being \dots including food, clothing, housing, medical care, necessary social services, and the right to security in the event of unemployment\dots” (United Nations, 1948)

          \item Jobs, Livelihood, Employment

          \item Food, Security, and Safety

          \item Equity — The “touchiest” of the principles; Could resources, natural, human, and economic, be distributed more evenly?

          \item Precautionary Principles — When a practice or product raises potentially significant threats to harm to human health or the environment, precautionary action should be taken to ban or restrict it

        \end{enumerate}

    \end{enumerate}

\end{document}

