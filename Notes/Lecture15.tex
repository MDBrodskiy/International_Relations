%%%%%%%%%%%%%%%%%%%%%%%%%%%%%%%%%%%%%%%%%%%%%%%%%%%%%%%%%%%%%%%%%%%%%%%%%%%%%%%%%%%%%%%%%%%%%%%%%%%%%%%%%%%%%%%%%%%%%%%%%%%%%%%%%%%%%%%%%%%%%%%%%%%%%%%%%%%%%%%%%%%%%%%%%%%%%%%%%%%%%%%%%%%%
% Written By Michael Brodskiy
% Class: International Relations
% Professor: J. Kropf
%%%%%%%%%%%%%%%%%%%%%%%%%%%%%%%%%%%%%%%%%%%%%%%%%%%%%%%%%%%%%%%%%%%%%%%%%%%%%%%%%%%%%%%%%%%%%%%%%%%%%%%%%%%%%%%%%%%%%%%%%%%%%%%%%%%%%%%%%%%%%%%%%%%%%%%%%%%%%%%%%%%%%%%%%%%%%%%%%%%%%%%%%%%%

\documentclass[12pt]{article} 
\usepackage{alphalph}
\usepackage[utf8]{inputenc}
\usepackage[russian,english]{babel}
\usepackage{titling}
\usepackage{amsmath}
\usepackage{graphicx}
\usepackage{enumitem}
\usepackage{amssymb}
\usepackage[super]{nth}
\usepackage{everysel}
\usepackage{ragged2e}
\usepackage{geometry}
\usepackage{fancyhdr}
\usepackage{cancel}
\usepackage{siunitx}
\usepackage{expl3}
\usepackage[version=4]{mhchem}
\usepackage{hpstatement}
\usepackage{xcolor}
\geometry{top=1.0in,bottom=1.0in,left=1.0in,right=1.0in}
\newcommand{\subtitle}[1]{%
  \posttitle{%
    \par\end{center}
    \begin{center}\large#1\end{center}
    \vskip0.5em}%

}
\usepackage{hyperref}
\hypersetup{
colorlinks=true,
linkcolor=blue,
filecolor=magenta,      
urlcolor=blue,
citecolor=blue,
}

\urlstyle{same}


\title{Lecture 15 Notes}
\date{July 20, 2021}
\author{Michael Brodskiy\\ \small Instructor: Prof. Kropf}

% Mathematical Operations:

% Sum: $$\sum_{n=a}^{b} f(x) $$
% Integral: $$\int_{lower}^{upper} f(x) dx$$
% Limit: $$\lim_{x\to\infty} f(x)$$

\begin{document}

    \maketitle

    \begin{enumerate}

      \item Tragedy of the Commons — Garrett Hardin’s metaphorical story of a village green where all can graze their sheep, but no one is accountable for its upkeep. The unregulated commons is eventually overgrazed and destroyed

      \item Ecosystem — A natural area whose plants, animals, and physical environment are closely interdependent. Each ecosystem has a carrying capacity, or limit on the amount of life it can sustain

      \item Population Explosion — The crisis that many ecologists point to as the central problem – too many people demanding more than the Earth can provide

      \item Malthusianism — Author Thomas Malthus suggests in 1798 that the world’s population would grow exponentially while its food grew arithmetically, leading to famine and death

      \item Sustainable Development — The major theme of the 1992 “Earth Summit” at Rio de Janeiro. Refers to economic growth that does not deplete resources and destroy ecosystems

      \item Core Values — the fundamental belief structures that influence human attitudes toward ecology

        \begin{enumerate}

          \item Anthropocentrism — Sacrificing species to satisfy human wants

          \item Contempocentrism — Lack of regard for future generations

        \end{enumerate}

      \item Amplifiers — The instrumental means by which human values, behaviors, and possession are extended or expanded

        \begin{enumerate}

          \item Population Growth — Impacts of a projected 9–14 billion people by 2050

          \item Technology — Unintended consequences of CFCs

        \end{enumerate}

      \item Consumptive Behavior — The tension between human needs and wants, and its ecological consequences as a function of material wealth. 

        \begin{enumerate}

          \item Poverty — Deforestation for fuel wood in developing countries

          \item Affluence — High per-capita consumption of “throw away” goods

        \end{enumerate}

      \item Political Economy — The dominant economic structure and ideology used to explain environmental problems

        \begin{enumerate}
            
          \item Market Failure — Unpriced costs of acid rain pollution and burning fossil fuels
            
          \item Failure to Have Markets — examples include overfishing as a “tragedy of the commons.”

        \end{enumerate}

      \item The following seven problems can be plausibly considered to contribute to conflict within and among countries:

        \begin{enumerate}

          \item Global Warming/Climate Change

          \item Ozone Depletion (Stratospheric)

          \item Acid Deposition (Acid Rain)

          \item Deforestation (Loss of Biodiversity)

          \item Degradation of Agricultural Land (desertification)

          \item Overuse and pollution of water

          \item Depletion of fish stocks 

        \end{enumerate}

      \item From these 7 problems, we can expect 4 principle social effects leading to international conflict:

        \begin{enumerate}

          \item Decreased agricultural production, leading to food shortages

          \item Economic decline

          \item Population displacement

          \item Disruption of legitimized and authoritative institutions and social relations

        \end{enumerate}

    \end{enumerate}

\end{document}

