%%%%%%%%%%%%%%%%%%%%%%%%%%%%%%%%%%%%%%%%%%%%%%%%%%%%%%%%%%%%%%%%%%%%%%%%%%%%%%%%%%%%%%%%%%%%%%%%%%%%%%%%%%%%%%%%%%%%%%%%%%%%%%%%%%%%%%%%%%%%%%%%%%%%%%%%%%%%%%%%%%%%%%%%%%%%%%%%%%%%%%%%%%%%
% Written By Michael Brodskiy
% Class: International Relations
% Professor: J. Kropf
%%%%%%%%%%%%%%%%%%%%%%%%%%%%%%%%%%%%%%%%%%%%%%%%%%%%%%%%%%%%%%%%%%%%%%%%%%%%%%%%%%%%%%%%%%%%%%%%%%%%%%%%%%%%%%%%%%%%%%%%%%%%%%%%%%%%%%%%%%%%%%%%%%%%%%%%%%%%%%%%%%%%%%%%%%%%%%%%%%%%%%%%%%%%

\documentclass[12pt]{article} 
\usepackage{alphalph}
\usepackage[utf8]{inputenc}
\usepackage[russian,english]{babel}
\usepackage{titling}
\usepackage{amsmath}
\usepackage{graphicx}
\usepackage{enumitem}
\usepackage{amssymb}
\usepackage[super]{nth}
\usepackage{everysel}
\usepackage{ragged2e}
\usepackage{geometry}
\usepackage{fancyhdr}
\usepackage{cancel}
\usepackage{siunitx}
\usepackage{expl3}
\usepackage[version=4]{mhchem}
\usepackage{hpstatement}
\usepackage{xcolor}
\geometry{top=1.0in,bottom=1.0in,left=1.0in,right=1.0in}
\newcommand{\subtitle}[1]{%
  \posttitle{%
    \par\end{center}
    \begin{center}\large#1\end{center}
    \vskip0.5em}%

}
\usepackage{hyperref}
\hypersetup{
colorlinks=true,
linkcolor=blue,
filecolor=magenta,      
urlcolor=blue,
citecolor=blue,
}

\urlstyle{same}


\title{Lecture 13 Notes}
\date{July 13, 2021}
\author{Michael Brodskiy\\ \small Instructor: Prof. Kropf}

% Mathematical Operations:

% Sum: $$\sum_{n=a}^{b} f(x) $$
% Integral: $$\int_{lower}^{upper} f(x) dx$$
% Limit: $$\lim_{x\to\infty} f(x)$$

\begin{document}

    \maketitle

    \begin{enumerate}

      \item Security Theory — Five Overlapping Approaches:

        \begin{enumerate}

          \item National Security — A perspective on security issues that looks out from a nation’s capital. The primary concern is the survival and well-being of the state. The threat or use of military power is viewed as the principle instrument used to ensure its survival

          \item International Security — A perspective on security issues which recognizes that the security of one state is interconnected with that of others. It views the collective use of military power as an important instrument of policy

          \item Regional Security — Takes the same perspective of international security, but focuses on other regions of the world, such as Latin America, Africa, or Asia. 

          \item Transstate Security — A new concept within security studies which asserts that, in the post-Cold War world, substate and transstate actors will constitute important sources of instability. These actors include ethno-national movements, religious extremists, criminal organizations, terrorists, and insurgents 

          \item Global Security — As a conceptual category it seeks to broaden the security agenda beyond the military and politico-military matters to include human rights, environmental protection, economic prosperity, and social development. It often carries the normative objective of replacing coercion, conflict, and war with cooperation, bargaining, and peaceful change

        \end{enumerate}

      \item International and Global Security in the Post-Cold War Era

        \begin{enumerate}

          \item Students of international politics deal with some of the most profound questions it is possible to consider. Amongst the most important of these is whether international security is possible to achieve in the kind of world in which we live. For much of the intellectual history of the subject a debate has raged about the causes of war

          \item Key Points to Consider

            \begin{enumerate}

              \item Security is a “contested concept”

              \item The meaning of security has been broadened to include political, economic, societal, and environmental, as well as military, aspects

              \item Differing arguments exist about the tension between national and international security

              \item Different views have emerged about the significance of 9/11 for the future of international security

              \item Debates about security have traditionally focused on the role of the state in IR. Realists emphasize the perennial problem of insecurity

              \item Realists point out the problem of “relative gains” whereby states compare their gains with those of other states when making decisions about security

              \item Neo-realists reject the significance of international institutions in helping many to achieve peace and security, while liberal institutionalists argue that institutions can provide a framework for cooperation which can help overcome the dangers of security competition between states

              \item Democratic peace theory emerged in the 1980s. The main argument was that the spread of democracy would lead to greater international security

              \item Democratic peace theory is based on Kantian logic emphasizing 3 elements: a. democratic representation, b. commitment to human rights, c. transnational interdependence

              \item Wars between democracies are seen as being rare and they are believed to settle mutual conflicts of interest without the threat or use of force more often than non-democratic states

              \item Collective security is based on 3 main conditions: a. states must renounce the use of military force to alter the status quo,  b. they must broaden their view of national interest to take in the interests of the international community,  c. states must overcome their fear and learn to trust each other

              \item Social constructivist thinkers base their ideas on two main assumptions:

                \begin{enumerate}

                  \item The fundamental structures of international politics are socially constructed

                  \item Changing the way we think about international relations can help to bring about greater international security

                \end{enumerate}

              \item Feminist writers argue that: a. gender tends to be left out of the literature on international security, despite the impact of war on women, b. bringing gender issues back in will result in a reconceptualization of the study of international security

              \item Post-modernists: a. aim to replace “realist discourse” with “communitarian discourse”, b. emphasize the importance of ideas and discourse in thinking about international security, c. try to reconceptualize the debate about global security by looking at questions which have been ignored by traditional approaches

              \item There are disputes about whether globalization will contribute to the weakening of the state or simply its transformation, and over whether a global society can be created which will usher in a new period of peace and security. Globalization itself appears to have an ambivalent impact on international security

              \item Different theorists have contrasting views about whether global security has changed fundamentally since September 11, 2001

            \end{enumerate}

        \end{enumerate}

      \item Peak Oil vs. Climate Change

        \begin{enumerate}

          \item Robert Hirsch was the lead author of a US Department of Energy report on peak oil in 2005. Hirsch was told never to talk about the report. This is what he has said recently: “\dots if you spend some time looking at peak oil, if you’re a reasonably intelligent person, you see that catastrophic things are going to happen to the world.  We’re talking about major damage, major change in our civilization. Chaos, economic disaster, wars, all kinds of things \dots really bad things.”

        \end{enumerate}

      \item What can be done by governments?

        \begin{enumerate}

          \item Introduce Cap \& Share:

            \begin{enumerate}

              \item Progressively cap fossil fuel use at point of entry to the economy (rather than \ce{CO2} emissions at multiple exit points). Controlled to give stable, high fossil fuel prices  

              \item Energy input permits to be given to all adults

              \item Setting up an Exchange to sell these to energy importers or producers thus sharing scarcity rent with adult population and partially compensating for rising prices

              \item Tax on permit purchase provides funds for mitigation and adaptation

            \end{enumerate}

          \item Legislation and fiscal policy to:

            \begin{enumerate}

              \item Stop fossil fuel subsidies

              \item Introduce Cap \& Share to cap fossil fuel use and share some of the scarcity rent with the general population

              \item Discourage the second dash for gas

              \item Decarbonise electricity production

              \item Pursue growth of the green sector by stimulating investment in renewables \& energy efficiency/conservation

              \item Reduce material use in construction, manufacture, packing, etc

              \item Redistribute wealth to mitigate against social unrest

              \item Set tighter carbon budgets

              \item Require LA’s to set and publish GHG emission budgets and descent plans and report on compliance with these

              \item Tackle non\ce{CO2} GHG emissions

              \item Protect and improve carbon sinks (woodland and pasture)

              \item Facilitate transition to a steady state economy operating within ecological limits

              \item Encourage decentralisation and local resilience

            \end{enumerate}

        \end{enumerate}

      \item Final Summary

        \begin{enumerate}
            
          \item In the runaway climate change scenario, peak oil and economic decline is delayed by burning substantially all oil, gas and coal reserves without CCS, global mean temp could increase by as much as $15^{\circ}$C. The likely consequences are horrendous

          \item The PO catastrophe scenario could limit climate change by precipitating chaotic irreversible economic collapse. If this happened quickly and drastically it would be beneficial for biodiversity but could be unspeakably dreadful for civilisation. Controlled economic descent is needed instead

        \end{enumerate}

    \end{enumerate}

\end{document}

