%%%%%%%%%%%%%%%%%%%%%%%%%%%%%%%%%%%%%%%%%%%%%%%%%%%%%%%%%%%%%%%%%%%%%%%%%%%%%%%%%%%%%%%%%%%%%%%%%%%%%%%%%%%%%%%%%%%%%%%%%%%%%%%%%%%%%%%%%%%%%%%%%%%%%%%%%%%%%%%%%%%%%%%%%%%%%%%%%%%%%%%%%%%%
% Written By Michael Brodskiy
% Class: International Relations
% Professor: J. Kropf
%%%%%%%%%%%%%%%%%%%%%%%%%%%%%%%%%%%%%%%%%%%%%%%%%%%%%%%%%%%%%%%%%%%%%%%%%%%%%%%%%%%%%%%%%%%%%%%%%%%%%%%%%%%%%%%%%%%%%%%%%%%%%%%%%%%%%%%%%%%%%%%%%%%%%%%%%%%%%%%%%%%%%%%%%%%%%%%%%%%%%%%%%%%%

\documentclass[12pt]{article} 
\usepackage{alphalph}
\usepackage[utf8]{inputenc}
\usepackage[russian,english]{babel}
\usepackage{titling}
\usepackage{amsmath}
\usepackage{graphicx}
\usepackage{enumitem}
\usepackage{amssymb}
\usepackage[super]{nth}
\usepackage{everysel}
\usepackage{ragged2e}
\usepackage{geometry}
\usepackage{fancyhdr}
\usepackage{cancel}
\usepackage{siunitx}
\usepackage{xcolor}
\geometry{top=1.0in,bottom=1.0in,left=1.0in,right=1.0in}
\newcommand{\subtitle}[1]{%
  \posttitle{%
    \par\end{center}
    \begin{center}\large#1\end{center}
    \vskip0.5em}%

}
\usepackage{hyperref}
\hypersetup{
colorlinks=true,
linkcolor=blue,
filecolor=magenta,      
urlcolor=blue,
citecolor=blue,
}

\urlstyle{same}


\title{Lecture 6 Notes}
\date{June 22, 2021}
\author{Michael Brodskiy\\ \small Instructor: Prof. Kropf}

% Mathematical Operations:

% Sum: $$\sum_{n=a}^{b} f(x) $$
% Integral: $$\int_{lower}^{upper} f(x) dx$$
% Limit: $$\lim_{x\to\infty} f(x)$$

\begin{document}

    \maketitle

    \begin{enumerate}

      \item Postmodernism — The fourth and newest system-level theory of IR is known as postmodernism. This paradigm incorporates a rather diverse array of theories developed over the last 25 years, including constructivism, feminism, critical theory, and poststructuralism

      \item Although these theories offer distinct views of IR, they are united by their rejection of empirically based positivist traditions that are the cornerstone of the first three theories of IR

      \item At the root of the argument between postmodernism and mainstream IR theory is a difference over epistemology, which is the study of the origin, nature, and limits of human knowledge

      \item Traditional theorists of IR (realists, liberals, class system theorists) believe that it is possible to acquire objective knowledge of the world

      \item Postmodernists reject positivism, which is a philosophical movement characterized by an emphasis upon science and scientific method as the only dependable sources of knowledge

      \item Postmodernists question the validity of positivist epistemology. Thus, the postmodernist critique of conventional IR theory can be divided into three basic parts:

        \begin{enumerate}

          \item The first part involves hermeneutics and its effect on theory development. Hermeneutics is the study of how we interpret, explain, and draw meaning from language

          \item Secondly, many postmodernists argue that there is no such thing as objective, value-free theory. Theory is merely the product of our perception of reality

          \item Third, postmodernists' uncertainty about the objectivity of language and the ability of theorists to produce value-neutral theory caused them to search for alternative ways to interpret IR theory

        \end{enumerate}

      \item Deconstruction is a theory about language and literature that postmodernists adapted to analyze theories of IR. It is an analytical method that seeks to take apart, or “deconstruct” verbal and written language in search of hidden meanings implanted inside (sometimes called “reversing hierarchies”)

      \item The Two Main Branches of Postmodernism:

        \begin{enumerate}

          \item Constructivism — The first branch of postmodernism is called constructivism, a theory based on the claim that our understanding of reality is socially “constructed.” Constructivists place great attention on the role of ideas and beliefs in shaping our understanding of the world

            \begin{enumerate}

              \item Social construction is the study of how people's identities, values, and ideas are defined by their group affiliations

            \end{enumerate}

          \item Feminism — A key component of feminist theory is its questioning of the validity of the distribution of power, a key component in the behavior of and relations between states as the system currently stands. Feminist postmodernists emphasize the social construction of gender roles and their impact on the structure of society and international relations

            \begin{enumerate}

              \item Feminist theorists often use the term androcentric (male-centered) to describe the idea that mainstream theories of IR (particularly realism) ignore alternative viewpoints and, instead, rely on essentially masculine interpretations of world affairs

            \end{enumerate}

        \end{enumerate}

      \item Components of Postmodernism:

        \begin{enumerate}

          \item Focus of Analysis — Social construction and impact of ideas, discourse, group identities, and gender on international relations

          \item View of Human Nature — A product of cultural and social circumstances; Constantly changing and evolving

          \item Behavior of States — Driven by beliefs, norms, and social identities

          \item Goals of Postmodernism — Broaden the agenda of international relations theory; Expose the power relationships of contemporary discourse in global politics 

        \end{enumerate}

      \item Key Concepts

        \begin{enumerate}

          \item Androcentric — Describes the idea that traditional theories of international relations (particularly realism) ignore alternate viewpoints and, instead, rely on essentially masculine interpretations of world affairs

          \item Constructivism — A postmodern theory of international relations that is based on the claim that our understanding of reality is socially constructed. Constructivists place great attention on the role of ideas and belief in shaping our understanding of the world. They emphasize the identities and interests of individuals and states and the ways in which those preferences are socially constructed

          \item Deconstruction — A theory about language and literature developed in the 1970s that postmodernists have adapted to analyze theories of international relations. Deconstruction is an analytical method that seeks to take apart or “deconstruct” verbal and written language in search of the hidden meanings inside

          \item Epistemology — The study of the origin, nature, and limits of human knowledge

          \item Gender/Gender Roles — Encompasses the social and cultural distinctions, as well as differences in traditional roles, between the two sexes, not simply clinical of biological classifications. The jobs, tasks, and activities that are traditionally associated to either men or women as a group

          \item Hermeneutics — The study of her we interpret, explain, and draw meaning from language

          \item Positivism — A philosophical movement characterized by an emphasis upon science and scientific method as the only dependable sources of knowledge. Positivists believe that reliable knowledge can only be acquired through experimental investigation and empirical observation

          \item Social Constructivism — The study of how people's identities, values, and ideas are developed by their group affiliations

          \item A hidden, underlying meaning or interpretation that can presumably be discerned by close examination of the words and phrases chosen by the author

        \end{enumerate}

      \item Critiques of Postmodernism and Feminism

        \begin{enumerate}

          \item Criticism of postmodernist theory centers on three major problems. First, postmodernism rejects the positivist underpinnings of mainstream IR theory, but doesn't present any clear alternatives with which to replace it

          \item Second, postmodernism is merely a critique of traditional IR theory, and nothing more

          \item Third, many mainstream IR theorists question the intellectual rigor of postmodernists who shun mainstream social science methodology and rely solely on non-falsifiable assertions regarding the nature and course of international relations

          \item Finally, the postmodernist emphasis on the social construction of reality and the lack of objective truths leads many to wonder whether postmodernism is simply an anti-intellectual exercise that leads us nowhere

          \item The Alan Sokal Affair\dots

          \item A major criticism of the feminist paradigm is its failure to provide a comprehensive theoretical construct for analyzing international relations

          \item As a prescriptive theory, feminism falls into the trap of focusing too much of its efforts on how the situation in world politics, and the study of world politics, might be change

          \item Are feminists guilty of relying on their own stereotypes of gender characteristics? By using selective characterizations — women are more cooperative and peaceful, while men are more violent and aggressive — feminist theorists risk reinforcing the same gender stereotyping they are trying to overcome

          \item Larger forces (human nature, disparities of wealth and power, anarchic world system, to name just a few) shape behavior of various actors on the world stage irrespective of gender

        \end{enumerate}

    \end{enumerate}

\end{document}

