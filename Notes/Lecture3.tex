%%%%%%%%%%%%%%%%%%%%%%%%%%%%%%%%%%%%%%%%%%%%%%%%%%%%%%%%%%%%%%%%%%%%%%%%%%%%%%%%%%%%%%%%%%%%%%%%%%%%%%%%%%%%%%%%%%%%%%%%%%%%%%%%%%%%%%%%%%%%%%%%%%%%%%%%%%%%%%%%%%%%%%%%%%%%%%%%%%%%%%%%%%%%
% Written By Michael Brodskiy
% Class: International Relations
% Professor: J. Kropf
%%%%%%%%%%%%%%%%%%%%%%%%%%%%%%%%%%%%%%%%%%%%%%%%%%%%%%%%%%%%%%%%%%%%%%%%%%%%%%%%%%%%%%%%%%%%%%%%%%%%%%%%%%%%%%%%%%%%%%%%%%%%%%%%%%%%%%%%%%%%%%%%%%%%%%%%%%%%%%%%%%%%%%%%%%%%%%%%%%%%%%%%%%%%

\documentclass[12pt]{article} 
\usepackage{alphalph}
\usepackage[utf8]{inputenc}
\usepackage[russian,english]{babel}
\usepackage{titling}
\usepackage{amsmath}
\usepackage{graphicx}
\usepackage{enumitem}
\usepackage{amssymb}
\usepackage[super]{nth}
\usepackage{everysel}
\usepackage{ragged2e}
\usepackage{geometry}
\usepackage{fancyhdr}
\usepackage{cancel}
\usepackage{siunitx}
\usepackage{xcolor}
\geometry{top=1.0in,bottom=1.0in,left=1.0in,right=1.0in}
\newcommand{\subtitle}[1]{%
  \posttitle{%
    \par\end{center}
    \begin{center}\large#1\end{center}
    \vskip0.5em}%

}
\usepackage{hyperref}
\hypersetup{
colorlinks=true,
linkcolor=blue,
filecolor=magenta,      
urlcolor=blue,
citecolor=blue,
}

\urlstyle{same}


\title{Lecture 3 Notes}
\date{June 17, 2021}
\author{Michael Brodskiy\\ \small Instructor: Prof. Kropf}

% Mathematical Operations:

% Sum: $$\sum_{n=a}^{b} f(x) $$
% Integral: $$\int_{lower}^{upper} f(x) dx$$
% Limit: $$\lim_{x\to\infty} f(x)$$

\begin{document}

    \maketitle

    \begin{enumerate}

      \item Realism — The oldest theory for understanding and explaining international politics, the roots of which extend back 2,500 years

      \item The fundamental principles and implications of realism can be found in the writings of the ancient Greek historian Thucydides and the Italian Renaissance political philosopher Niccolo Machiavelli

      \item Scholars argue that leaders rely on realist theory because it presents a “realistic” view of international relations and focuses on how the world \textit{is} rather than how it \textit{ought to be}

      \item For realists, power is the key factor in understanding international relations

      \item Global politics is considered a contest for power among states

      \item A state's power is measured primarily in terms of its military capabilities

      \item The Components of Realism:

        \begin{enumerate}

          \item Focus of Analysis — Struggle for power among states in an anarchic international system

          \item Major Actors — States

          \item Behavior of States — Rational, unitary actors

          \item Goals of States — Enhance power and security

          \item View of Human Nature — Pessimistic

          \item Condition of the International System — Anarchic; Self-help system

        \end{enumerate}

      \item Key Concepts

        \begin{enumerate}

          \item Security Dilemma — The problem of fear, insecurity, and lack of trust among states living in an anarchic international system. A state builds up arms to protect itself, but that just makes other states fearful, creating an arms race, and making the world less secure

          \item Balance of Power — A policy aimed at maintaining the international status quo. Peace and security are best preserved when power is distributed among five or more states, and no single state has a preponderance of power

          \item Power Politics — Policies in which force, or the threat of force, is the primary method used to further a state's interests. For realists, international relations is a struggle for power and security among competing states

          \item Anarchy — Refers to the lack of a central authority or government to enforce law and order between states throughout the globe

          \item Self-help System — A neo-realist concept that, in an anarchic international system, where there is no overarching global authority (like a world government) to enforce peace and stability, each state is responsible for its survival and can not rely on other states

          \item Rational Actor — Refers to the realist assumption that states generally pursue attainable, prudent goals that commensurate with their power or capability to achieve

          \item Hegemon (Hegemony) — A state with overwhelming military, economic, and political power that has the ability to maintain its dominant position in the international system

          \item Neo-Realism — A variant of realism that contends that the struggle for power among states is the result of the anarchic structure of the international system as a whole, rather than some fundamental aspect of human nature

        \end{enumerate}

      \item Critiques of Realist Theory

        \begin{enumerate}

          \item In the contemporary world, can war be considered a natural extension of politics among nations, since nuclear weapons have made the pursuit of power using war an essentially unwinnable endeavor?

          \item Realist theory tends to ignore the current expansion of cooperation between states

          \item States are no longer the only important actors on the international stage

          \item The increasing relevance of sub-state actors, such as terrorists. Realists contend that states are primary actors on the international stage and that all other identities are less important to our understanding of global affairs

          \item A final weakness is the theory's inability to account for peaceful change

        \end{enumerate}

    \end{enumerate}

\end{document}

