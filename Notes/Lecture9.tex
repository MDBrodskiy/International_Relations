%%%%%%%%%%%%%%%%%%%%%%%%%%%%%%%%%%%%%%%%%%%%%%%%%%%%%%%%%%%%%%%%%%%%%%%%%%%%%%%%%%%%%%%%%%%%%%%%%%%%%%%%%%%%%%%%%%%%%%%%%%%%%%%%%%%%%%%%%%%%%%%%%%%%%%%%%%%%%%%%%%%%%%%%%%%%%%%%%%%%%%%%%%%%
% Written By Michael Brodskiy
% Class: International Relations
% Professor: J. Kropf
%%%%%%%%%%%%%%%%%%%%%%%%%%%%%%%%%%%%%%%%%%%%%%%%%%%%%%%%%%%%%%%%%%%%%%%%%%%%%%%%%%%%%%%%%%%%%%%%%%%%%%%%%%%%%%%%%%%%%%%%%%%%%%%%%%%%%%%%%%%%%%%%%%%%%%%%%%%%%%%%%%%%%%%%%%%%%%%%%%%%%%%%%%%%

\documentclass[12pt]{article} 
\usepackage{alphalph}
\usepackage[utf8]{inputenc}
\usepackage[russian,english]{babel}
\usepackage{titling}
\usepackage{amsmath}
\usepackage{graphicx}
\usepackage{enumitem}
\usepackage{amssymb}
\usepackage[super]{nth}
\usepackage{everysel}
\usepackage{ragged2e}
\usepackage{geometry}
\usepackage{fancyhdr}
\usepackage{cancel}
\usepackage{siunitx}
\usepackage{xcolor}
\geometry{top=1.0in,bottom=1.0in,left=1.0in,right=1.0in}
\newcommand{\subtitle}[1]{%
  \posttitle{%
    \par\end{center}
    \begin{center}\large#1\end{center}
    \vskip0.5em}%

}
\usepackage{hyperref}
\hypersetup{
colorlinks=true,
linkcolor=blue,
filecolor=magenta,      
urlcolor=blue,
citecolor=blue,
}

\urlstyle{same}


\title{Lecture 9 Notes}
\date{July 1, 2021}
\author{Michael Brodskiy\\ \small Instructor: Prof. Kropf}

% Mathematical Operations:

% Sum: $$\sum_{n=a}^{b} f(x) $$
% Integral: $$\int_{lower}^{upper} f(x) dx$$
% Limit: $$\lim_{x\to\infty} f(x)$$

\begin{document}

    \maketitle

    \begin{enumerate}

      \item Peace Studies Theory — Peace studies is a relatively new approach to the study of international relations and pushes boldly past may of the guiding principles that commonly characterize the study of foreign policy and the behavior of states

      \item Peace studies theorists argue that the study of international affairs should reach beyond the more traditional evaluations and measurements of power, balance of power, and national security

      \item They argue that in our interconnected world, issues such as poverty, social injustice, and environmental destruction, reach to the heart of our security as human beings

      \item However, in order for that peaceful change to occur at an international level, it must begin with a personal transformation. A personal transformation is a change or shift in an individual’s outlook, habits, or worldview in a way that makes that individual more socially conscious of the global effect of his or her actions

      \item Discussions of peace studies theory must begin with an examination of the impact of Gandhi. He remains the preeminent contemporary example of the nonviolent approach to resolving conflict

      \item Gandhi was a pacifist who sought political change through non- violent means.  Pacifism is the rejection of the use of force for dealing with domestic or international conflicts

      \item Though the study of international relations, with its theories and levels of analysis, generally falls in the realm of political science, peace studies is actually an interdisciplinary field

      \item This intellectual crossover is one of several key characteristics that distinguish peace studies theory from other paradigms found in the study of international affairs

      \item In addition to the inclusive approach of peace studies theory, this paradigm is also a prescriptive, rather than descriptive paradigm

      \item Peace studies is also considered to be a value-based theory.  A value is an ideal or principle that people generally consider worthwhile and desirable

      \item Students of I.R. should also understand that the ideal end of peace studies is not simply peace but positive peace. Positive peace is the absence of war in combination with the establishment of broader, worldwide forms of social justice, economic prosperity, and political power-sharing

      \item The idea of positive peace is contrasted with what some peace studies theorists call “negative peace,” Negative peace is defined as only the absence of war, a condition in which direct forms organized violence are absent but the underlying reasons for war, such as social injustice and economic exploitation, are left unresolved

      \item It is important to point out that advocates of peace studies differentiate between violence and conflict

      \item Conflict — as opposed to violence — is viewed by peace studies theorists as healthy debate, disagreement, or dialogue and an integral part of life and growth in human existence

      \item In peace studies, scientists and researchers are asked to look at the consequences (environmental, social, economic) of their work. The invention of the atomic bomb, chemical weaponry, and other such devices cannot be disassociated from their violent purpose

      \item Peace studies encourages scientists, scholars, and social activists from all fields to work together in assessing the potential political, social, and environmental ramifications of these instruments, as well as future technologies and innovations

      \item The variety of work under the rubric of peace studies seeks not only to understand the many causes of war and conflict but also to go beyond them in its quest for possibilities of peace building

      \item Components

        \begin{enumerate}

          \item Focus of Analysis — Promoting principles of positive peace

          \item Major Actors — Individuals; Individuals working through groups

          \item Approach of Peace Studies — Interdisciplinary; Value-oriented

          \item Goals of Peace Studies — Social justice, human rights, economic opportunity, and political equality; Building bridges between nations through person-to-person contact

        \end{enumerate}

      \item Key Concepts

        \begin{enumerate}

          \item Negative Peace — The absence of war. Direct forms of organized violence are absent, but the underlying reasons for war, such as social injustice and economic exploitation, are left unresolved. This term is contrasted with what peace studies theorists refer to as positive peace

          \item Pacifism — The rejection of the use of force for dealing with domestic or international conflicts

          \item Personal Transformation — A change or shift in an individual's nature, outlook, habits, or worldview in a way that makes that individual more socially conscious of the global effect of his or her actions

          \item Positive Peace — The absence of war in combination with the establishment of broader, worldwide forms of social justice, economic prosperity, and political power-sharing. This notion is contrasted with what peace studies theorists call negative peace

          \item Value — An ideal or principle that people generally consider worthwhile and desirable. Peace studies assigns value to the actions, policies, activities, and methods of governments and individuals based on their benefit or harm to society

        \end{enumerate}

      \item Critiques of Peace Studies Theory

        \begin{enumerate}

          \item Critics of peace studies can legitimately question its idealistic worldview, which some might even call utopian

          \item Critics add that peace studies theorists have developed such an optimistic outlook on the international system, that creating policies around it could be potentially dangerous in real-world situations

          \item Though noble in intent, the policies fostered by peace studies could subtly weaken the resolve of people and states to resist and combat a nation bent on extending its power or influence through aggressive actions

          \item On a theoretical level, peace studies has been similarly criticized for underestimating the inherent conflict between nation-states

          \item These critics argue that as a theoretical construct, peace studies theory is disjointed and lacking a coherent framework for analyzing international relations

        \end{enumerate}

    \end{enumerate}

\end{document}

