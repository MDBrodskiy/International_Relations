%%%%%%%%%%%%%%%%%%%%%%%%%%%%%%%%%%%%%%%%%%%%%%%%%%%%%%%%%%%%%%%%%%%%%%%%%%%%%%%%%%%%%%%%%%%%%%%%%%%%%%%%%%%%%%%%%%%%%%%%%%%%%%%%%%%%%%%%%%%%%%%%%%%%%%%%%%%%%%%%%%%%%%%%%%%%%%%%%%%%%%%%%%%%
% Written By Michael Brodskiy
% Class: International Relations
% Professor: J. Kropf
%%%%%%%%%%%%%%%%%%%%%%%%%%%%%%%%%%%%%%%%%%%%%%%%%%%%%%%%%%%%%%%%%%%%%%%%%%%%%%%%%%%%%%%%%%%%%%%%%%%%%%%%%%%%%%%%%%%%%%%%%%%%%%%%%%%%%%%%%%%%%%%%%%%%%%%%%%%%%%%%%%%%%%%%%%%%%%%%%%%%%%%%%%%%

\documentclass[12pt]{article} 
\usepackage{alphalph}
\usepackage[utf8]{inputenc}
\usepackage[russian,english]{babel}
\usepackage{titling}
\usepackage{amsmath}
\usepackage{graphicx}
\usepackage{enumitem}
\usepackage{amssymb}
\usepackage[super]{nth}
\usepackage{everysel}
\usepackage{ragged2e}
\usepackage{geometry}
\usepackage{fancyhdr}
\usepackage{cancel}
\usepackage{siunitx}
\usepackage{xcolor}
\geometry{top=1.0in,bottom=1.0in,left=1.0in,right=1.0in}
\newcommand{\subtitle}[1]{%
  \posttitle{%
    \par\end{center}
    \begin{center}\large#1\end{center}
    \vskip0.5em}%

}
\usepackage{hyperref}
\hypersetup{
colorlinks=true,
linkcolor=blue,
filecolor=magenta,      
urlcolor=blue,
citecolor=blue,
}

\urlstyle{same}


\title{Lecture 5 Notes}
\date{June 22, 2021}
\author{Michael Brodskiy\\ \small Instructor: Prof. Kropf}

% Mathematical Operations:

% Sum: $$\sum_{n=a}^{b} f(x) $$
% Integral: $$\int_{lower}^{upper} f(x) dx$$
% Limit: $$\lim_{x\to\infty} f(x)$$

\begin{document}

    \maketitle

    \begin{enumerate}

      \item Class System Theory — Based on four important concepts:

        \begin{enumerate}

          \item First, proponents contend that economic factors are the driving force of international politics

          \item Second, class system theory focuses on the development of the capitalist world economy and how it creates and perpetuates uneven development between advanced capitalist states and poor, less developed states

          \item Third, theorists point to an international class structure in which the advanced industrialized states in the center of the world capitalist system dominate and exploit poorer states, occupying the periphery of the system

          \item Finally, transnational class coalitions represent the primary actors in international politics

        \end{enumerate}

      \item Unlike economic liberals, class system theorists emphasize the exploitative nature of international economic ties between states

      \item According to class system theory, the structure and process of international relations is largely the result of the struggle between rich and poor countries over the control and distribution of economic resources. The tension between rich and poor countries is often referred to as the North-South conflict

      \item War is the result of capitalist states attempting to increase their wealth and power through imperialist foreign policies — policies which strong capitalist states seek to exploit weaker, non-capitalist states

      \item Class system theorists acknowledge that states are important actors, but they also emphasize that the dominant class exerts significant influence and often controls government policymakers. Unlike realism, which holds that states pursue national security interests, class system theory argues that states act in accordance with the wishes of the dominant economic class within the state

      \item The idea of transnational class coalitions suggests that economic class form close ties across national boundaries

      \item Most class system theorists rely on the fundamental assumptions of marxism, and its critique of capitalism, as the intellectual basis for their theories about international relations

      \item According to Marx, the driving force of history is the struggle between economic classes. He termed this condition dialectical materialism. The dialectic suggests that history moves through stages — from feudalism to capitalism to socialism, and, finally, to communism

      \item Marx contends that capitalist society is divided into three basic classes:

        \begin{enumerate}

          \item The dominant class, made up of wealthy capitalists, owns and controls the means of production, that is, factories and land

          \item The second class consists largely of workers living in the cities and working in there capitalist-owned factories. Marx wrote that the workers, or proletariat, are nothing more than “wage slaves,” paid subsistence wages and exploited by factory owners whose sole motivation is greed

          \item The third class is made up of peasants who work the land. Capitalist land owners exploit peasant farm labor in return for allowing the peasantry to live on their land

        \end{enumerate}

      \item The division of society into these three classes represents the core of marxist theory, emphasizing the history of society as the history of class struggle, in which the oppressed worker and peasant classes attempt to free themselves from the domination of the wealthy capitalist class

      \item Classical imperialism is the policy of expanding a state's power and authority by conquering and controlling territories called colonies. According to John Hobson, imperialism is an extension of the capitalist search and competition for cheap labor and raw materials, and occurs when wealthy states exert political and economic control and influence over weaker nations or territories

      \item Components of Class System Theory:

        \begin{enumerate}

          \item Focus of Analysis — Capitalist world system

          \item Major Actors — States, transnational class coalitions, multinational corporations (MNCs), and international organizations

          \item Behavior of States — Class struggle, accumulation of wealth for the capitalist class

          \item Goals of States — Enhance wealth of the capitalist class

          \item View of Human Nature — Selfish and dominating, but reformable

          \item Condition of the International System — Domination of capitalist world system, cycle of exploitation and dependency

        \end{enumerate}

      \item Key Concepts

        \begin{enumerate}

          \item Capitalist World System — A concept of class system theory that focuses on the exploitative nature of the global spread of capitalism

          \item Class Struggle — The marxist theory that history is a story of struggle between economic classes in which the oppressed worker and peasant classes attempt to free themselves from the domination of the wealthy capitalist class

          \item Dependency Theory — A concept associated with class system theorists tat asserts that trade, foreign investment, and even foreign aid between advanced industrialized countries and poor, lesser developed states is inherently exploitative and works to the disadvantage of poor nations

          \item Dialectical Materialism — A theory developed by Karl Marx posting that history moves through stages — from feudalism to capitalism to socialism, and, finally, communism. The transition from one stage to the next is often prompted by the struggle between economic classes, as well as by the development and spread of technology

          \item Imperialism — The policy of expanding a state's power and authority by conquering and controlling territories, called colonies. Said to be the final form of capitalism by Lenin

          \item Neo-Imperialism — The process of the international system in which the advanced industrial states' control of exports and raw materials in developing nations is simply a more subtle form of domination than the previous imperialist practices of the European colonial empires

          \item North-South Conflict — A phrase used to characterize the tension between rich and poor countries. North represents the wealthy industrialized states that lie primarily in the northern hemisphere, while South is the term used for less developed countries, mainly located in the southern hemisphere

          \item Proletariat — A marxist term for the industrial workers living in urban areas and working in capitalist-owned factories. Marx wrote that the workers are nothing more than “wage slaves,” paid subsistence wages and exploited by capitalist factory owners whose sole motivation is greed

          \item Transnational Class Coalitions — A concept of class system theory contending that economic classes form close ties across national boundaries. Unlike liberals who focus on the positive aspects of increasing international economic linkages, class system theorists emphasize the exploitative nature of the global economic system and the role that transnational actors, like multinational corporations, play in this exploitation

          \item Uneven Development — The propensity of capitalism to create and perpetuate an unequal dispersal of global wealth and prosperity

        \end{enumerate}

      \item Critiques of Class System Theory

        \begin{enumerate}

          \item Many critics argue that class system theory exaggerates the role of the world capitalist system in limiting development and ignores the impact that policies adopted in the less developed states have on their own economic development

          \item The theory also fails to address the impact that different development strategies have had on varies countries facing similar problems with economic development. Dependency theory can account only for those countries that have failed to develop

          \item Some class system theorists have been criticized for being ahistorical — a potential trap for any theorist of IR

          \item Class system theory also falls short in explaining the failure of marxist principles and communist doctrine in the former USSR and Eastern Bloc nations, as well as the capitalist reforms now under way in these countries

        \end{enumerate}

    \end{enumerate}

\end{document}

