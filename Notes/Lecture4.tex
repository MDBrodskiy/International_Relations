%%%%%%%%%%%%%%%%%%%%%%%%%%%%%%%%%%%%%%%%%%%%%%%%%%%%%%%%%%%%%%%%%%%%%%%%%%%%%%%%%%%%%%%%%%%%%%%%%%%%%%%%%%%%%%%%%%%%%%%%%%%%%%%%%%%%%%%%%%%%%%%%%%%%%%%%%%%%%%%%%%%%%%%%%%%%%%%%%%%%%%%%%%%%
% Written By Michael Brodskiy
% Class: International Relations
% Professor: J. Kropf
%%%%%%%%%%%%%%%%%%%%%%%%%%%%%%%%%%%%%%%%%%%%%%%%%%%%%%%%%%%%%%%%%%%%%%%%%%%%%%%%%%%%%%%%%%%%%%%%%%%%%%%%%%%%%%%%%%%%%%%%%%%%%%%%%%%%%%%%%%%%%%%%%%%%%%%%%%%%%%%%%%%%%%%%%%%%%%%%%%%%%%%%%%%%

\documentclass[12pt]{article} 
\usepackage{alphalph}
\usepackage[utf8]{inputenc}
\usepackage[russian,english]{babel}
\usepackage{titling}
\usepackage{amsmath}
\usepackage{graphicx}
\usepackage{enumitem}
\usepackage{amssymb}
\usepackage[super]{nth}
\usepackage{everysel}
\usepackage{ragged2e}
\usepackage{geometry}
\usepackage{fancyhdr}
\usepackage{cancel}
\usepackage{siunitx}
\usepackage{xcolor}
\geometry{top=1.0in,bottom=1.0in,left=1.0in,right=1.0in}
\newcommand{\subtitle}[1]{%
  \posttitle{%
    \par\end{center}
    \begin{center}\large#1\end{center}
    \vskip0.5em}%

}
\usepackage{hyperref}
\hypersetup{
colorlinks=true,
linkcolor=blue,
filecolor=magenta,      
urlcolor=blue,
citecolor=blue,
}

\urlstyle{same}


\title{Lecture 4 Notes}
\date{June 17, 2021}
\author{Michael Brodskiy\\ \small Instructor: Prof. Kropf}

% Mathematical Operations:

% Sum: $$\sum_{n=a}^{b} f(x) $$
% Integral: $$\int_{lower}^{upper} f(x) dx$$
% Limit: $$\lim_{x\to\infty} f(x)$$

\begin{document}

    \maketitle

    \begin{enumerate}

      \item Liberalism — Advances the idea that states cooperate as much as, if not more than, they compete

      \item This cooperation, liberals assert, is more consistent than the realists' notion of national interest among limited numbers of states

      \item States cooperate because it is in their interest to do so, and prosperity and stability in the international system are a direct result of that cooperation

      \item Liberals also believe that states are not motivated solely by national interest defined in terms of power

      \item While the “high politics” of national security and military power remain important, liberals maintain that economic, social, and environmental issues — or “low politics” — have become priorities on the international agenda

      \item For liberals, the establishment and success of the international order depends largely on four major factors:

        \begin{enumerate}

          \item The role of international institutions

          \item International rules and norms for behavior of states

          \item The increasing economic interdependence between nations

          \item Technological advancement and the growth of global communication

        \end{enumerate}

      \item Liberals believe that the transnational linkages that these four factors represent build incentives for cooperation, enhance trust between nations, and promote negotiation, rather than military confrontation as a means to resolve disputes between states

      \item Components of Liberalism:

        \begin{enumerate}

          \item Focus of Analysis — Enhancing global political and economic cooperation

          \item Major Actors — States, international organizations, multinational corporations (MNCs), and nongovernmental organizations (NGOs)

          \item Behavior of States — States not always rational factors; Compromise between various interest within the state

          \item Goals of States — Economic prosperity, international stability

          \item View of Human Nature — Optimistic

          \item Condition of the International System — Anarchic; But! Possible to mitigate anarchy

        \end{enumerate}

      \item Key Concepts

        \begin{enumerate}

          \item Interdependence — A concept that focuses on “mutual dependence” of nations in which two or more states are mutually sensitive and vulnerable to each other's actions. Economic liberals argue that this is a defining characteristic of our contemporary world

          \item International Law — The codification of rules that regulate the behavior of states and set limits upon what is permissible and what is not permissible. In theory, these rules are binding on states, as well as other international actors

          \item Collective Security — A liberal institutionalist concept of a system of world order in which aggression against an individual state is considered aggression against all states and will be met with a collective response from all states within the system

          \item Regime — The set of accepted rules, norms, and procedures that regulate the behavior of states and other actors in a given issue area

          \item Economic Liberalism — A theory of IR that highlights the economic transnational ties or linkages between states. With the merging of international and domestic economic interests, states have become increasingly interconnected or interdependent, and less dependent on, or less willing to use force or the threat of force to further their national interests

          \item Harmony of Interests — A liberal concept stating that the interest of all states coincides with the interest of each state. It implies that the incentive to cooperate with one another is stronger than the incentive for conflict

          \item Complex Interdependence — An economic liberalist concept that assumes states are not the only important actors, social welfare issues share center stage with security issues, and cooperation is as dominant a characteristic of IR as conflict

          \item Liberal Institutionalism — A theory of IR that contends that global cooperation is founded on three primary factors: 1) Enhancing the role and influence of international organizations, 2) Instituting collective security, and 3) Enforcing international law

          \item Multinational Corporations — MNCs are companies that have production facilities or branches in several countries

          \item Natural Law — A view that there is a system of rules, norms, and principles for the conduct of human affairs, founded on the belief that all people have basic, inalienable rights. In theory, these rights supersede any moral authority and cannot be legitimately denied by any government or society. Natural law is the philosophical foundation of international law

        \end{enumerate}

      \item Critiques of Liberalism

        \begin{enumerate}

          \item Critics of liberalism — both institutional and economic — contend that the theory places too much emphasis on the harmony of interests

          \item Liberalism underestimates the conflictual aspects of state interest and the benefits of cooperation can often be outweighed by fear and mistrust

          \item Liberals fail to take into account the powerful role of nationalism in world politics

          \item On issues of national security, the theme of national interest defined as self-interest (as opposed to liberalism's collective interest) is evident in many past and contemporary conflicts throughout the world

          \item Realists are skeptical of the liberals' notion that “low politics” have become as important as national security issues

          \item Finally, proponents of class system theory say that, in addition to being “Western-centric”, they view interdependence as exploitative of, rather than beneficial to, less-developed countries

        \end{enumerate}

    \end{enumerate}

\end{document}

