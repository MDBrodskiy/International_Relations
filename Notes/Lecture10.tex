%%%%%%%%%%%%%%%%%%%%%%%%%%%%%%%%%%%%%%%%%%%%%%%%%%%%%%%%%%%%%%%%%%%%%%%%%%%%%%%%%%%%%%%%%%%%%%%%%%%%%%%%%%%%%%%%%%%%%%%%%%%%%%%%%%%%%%%%%%%%%%%%%%%%%%%%%%%%%%%%%%%%%%%%%%%%%%%%%%%%%%%%%%%%
% Written By Michael Brodskiy
% Class: International Relations
% Professor: J. Kropf
%%%%%%%%%%%%%%%%%%%%%%%%%%%%%%%%%%%%%%%%%%%%%%%%%%%%%%%%%%%%%%%%%%%%%%%%%%%%%%%%%%%%%%%%%%%%%%%%%%%%%%%%%%%%%%%%%%%%%%%%%%%%%%%%%%%%%%%%%%%%%%%%%%%%%%%%%%%%%%%%%%%%%%%%%%%%%%%%%%%%%%%%%%%%

\documentclass[12pt]{article} 
\usepackage{alphalph}
\usepackage[utf8]{inputenc}
\usepackage[russian,english]{babel}
\usepackage{titling}
\usepackage{amsmath}
\usepackage{graphicx}
\usepackage{enumitem}
\usepackage{amssymb}
\usepackage[super]{nth}
\usepackage{everysel}
\usepackage{ragged2e}
\usepackage{geometry}
\usepackage{fancyhdr}
\usepackage{cancel}
\usepackage{siunitx}
\usepackage{xcolor}
\geometry{top=1.0in,bottom=1.0in,left=1.0in,right=1.0in}
\newcommand{\subtitle}[1]{%
  \posttitle{%
    \par\end{center}
    \begin{center}\large#1\end{center}
    \vskip0.5em}%

}
\usepackage{hyperref}
\hypersetup{
colorlinks=true,
linkcolor=blue,
filecolor=magenta,      
urlcolor=blue,
citecolor=blue,
}

\urlstyle{same}


\title{Lecture 10 Notes}
\date{July 6, 2021}
\author{Michael Brodskiy\\ \small Instructor: Prof. Kropf}

% Mathematical Operations:

% Sum: $$\sum_{n=a}^{b} f(x) $$
% Integral: $$\int_{lower}^{upper} f(x) dx$$
% Limit: $$\lim_{x\to\infty} f(x)$$

\begin{document}

    \maketitle

    \begin{enumerate}

      \item Foreign Policy — A strategy or planned course of action by decision-makers of a state, which aims to achieve specific goals defined in terms of national interest. Major steps include:

        \begin{enumerate}

          \item Translating national interest into specific goals/objectives

          \item Determining the national and domestic situational factors related to policy goals

          \item Analyzing the state's capabilities for achieving desired results

          \item Developing a plan or strategy to link capabilities with goals

          \item Undertaking the requisite actions

          \item Periodically reviewing and evaluating progress toward achievement of the desired results

        \end{enumerate}

      \item Foreign policy actions are difficult to evaluate because:

        \begin{enumerate}

          \item Short-range advantages and disadvantages must be weighed in relation to long-term consequences

          \item Their impact on other nations is difficult to evaluate

          \item Most policies result in a mixture of successes and failures that are hard to disentangle

        \end{enumerate}

      \item Foreign Policy Approaches

        \begin{enumerate}

          \item Realist/Idealist Dichotomy — Alternative approaches in forming foreign policy

            \begin{enumerate}

              \item Realist — Fundamentally empirical and pragmatic

              \item Idealist — Abstract principles involving international norms, legal codes, and moral/ethical values

            \end{enumerate}

          \item Revisionist — Foreign policy which seeks to alter the existing territorial, ideological, or power distribution to its advantage (expansionist and acquisitive)

          \item Status Quo — Foreign policy which seeks to maintain revisionists (conservative and “defensive”)

        \end{enumerate}

      \item Foreign Policy Components

        \begin{enumerate}

          \item Objectives

          \item Situational factors

          \item National interest — The fundamental objective and ultimate determinant that guides decision-makers of a state in making foreign policy. There are 5 components: 

            \begin{enumerate}

              \item National Security (preemptive vs. preventive warfare)

              \item Free Trade/Free Markets (capitalism)

              \item Democracy

              \item World Peace

              \item Humanitarian Concerns

            \end{enumerate}

        \end{enumerate}

      \item Foreign Policy Process

        \begin{enumerate}

          \item Capability analysis

          \item Intelligence

          \item “Groupthink”

          \item Decision-makers — Those individuals who exercise the powers of making and implementing foreign policy decisions

            \begin{enumerate}

              \item Opinion elites

              \item General public

              \item Cabinet secretaries

              \item Foreign policy bureaucracy

            \end{enumerate}

        \end{enumerate}

      \item American Foreign Policy — Common Themes and Historical Concepts

        \begin{enumerate}

          \item The Monroe Doctrine

          \item Isolationism and Internationalism

          \item Dollar Diplomacy (Roosevelt, Taft, Wilson)

          \item Good Neighbor Policy (FDR)

          \item The Marshall Plan

          \item The Truman Doctrine (Containment)

          \item Alliance for Progress (JFK)

          \item Agency for International Development (AID)

          \item The Military-Industrial Complex

        \end{enumerate}

      \item Realism vs. Idealism in Foreign Policy

        \begin{enumerate}

          \item Realism — Design policy based on “what is”

            \begin{enumerate}

              \item Outlook — Isolationist

              \item Power — Utilize “hard” power

              \item Leadership — Unilateral

              \item Defense — Large; National Missile Defense; 2-War

              \item Arms and Weapons — Tension $\rightarrow$ Arms $\rightarrow$ War

              \item Foreign Aid — Lower; Focus on Military

              \item Democracy, Human Rights, Environment, United Nations — Not As Important

              \item Trade and Business — Will not necessarily promote peace

            \end{enumerate}

          \item Idealism — Design policy based on how the world “ought to be”

            \begin{enumerate}

              \item Outlook — Internationalist

              \item Power — Utilize “soft” power

              \item Leadership — Multilateral

              \item Defense — Smaller; Use weapons we already have

              \item Arms and Weapons — Arms $\rightarrow$ Tension $\rightarrow$ War

              \item Foreign Aid — Higher; Focus on Social, Economic

              \item Democracy, Human Rights, Environment, United Nations — More Important

              \item Trade and Business — Will help promote peace

            \end{enumerate}

        \end{enumerate}

      \item Three Schools of Though on America's Future

        \begin{enumerate}

          \item Declinism — One side in the persistent “debate” about the future of American power and influence. Declinists believe that the relative power position of the US is waning

          \item American Exceptionalism — The other side of the debate on the future of America's power. They believe that America is unique in world history, and thus will continue to grow in power and influence

          \item Neoimperialism — An alternative to both theories. This theory suggests that, while American leaders focus on global leadership, they are ignoring pressing social, economic, and political problems at home

        \end{enumerate}

      \item Walter Russell Mead's Four Schools of American Foreign Policy

        \begin{enumerate}

          \item Jacksonian

            \begin{enumerate}

              \item First Priority — Physical security and economic well-being of the American populace

              \item US should not seek out foreign quarrels but should fight to win if war starts

              \item Values — Self-reliance above all

              \item Jacksonian Presidents — Reagan; Bush (II)

            \end{enumerate}

          \item Hamiltonian

            \begin{enumerate}

              \item First Priority — Economic primacy of the US (mercantilism)

              \item The relationship between government and big business is key to survival and success of a country

              \item Legacies — IMF, World Bank, NAFTA, WTO

              \item Hamiltonian Presidents — Bush (I); Clinton

            \end{enumerate}

          \item Jeffersonian

            \begin{enumerate}

              \item First Priority — Protection of American democracy on the home front

              \item Foreign entanglements always bad for democratic systems and highly skeptical of projects that involve the US abroad

              \item Legacies — ACLU

              \item Jeffersonian Presidents — None in the \nth{20} century

            \end{enumerate}

          \item Wilsonian

            \begin{enumerate}

              \item First Priority — Spreading American democratic and social values throughout the world

              \item US should be involved in the world with a peaceful international community based on the rule of law

              \item Legacies — The United Nations

              \item Wilsonian Presidents — McKinley; Carter

            \end{enumerate}
                        
        \end{enumerate}

    \end{enumerate}

\end{document}

