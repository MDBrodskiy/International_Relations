%%%%%%%%%%%%%%%%%%%%%%%%%%%%%%%%%%%%%%%%%%%%%%%%%%%%%%%%%%%%%%%%%%%%%%%%%%%%%%%%%%%%%%%%%%%%%%%%%%%%%%%%%%%%%%%%%%%%%%%%%%%%%%%%%%%%%%%%%%%%%%%%%%%%%%%%%%%%%%%%%%%%%%%%%%%%%%%%%%%%%%%%%%%%
% Written By Michael Brodskiy
% Class: International Relations
% Professor: J. Kropf
%%%%%%%%%%%%%%%%%%%%%%%%%%%%%%%%%%%%%%%%%%%%%%%%%%%%%%%%%%%%%%%%%%%%%%%%%%%%%%%%%%%%%%%%%%%%%%%%%%%%%%%%%%%%%%%%%%%%%%%%%%%%%%%%%%%%%%%%%%%%%%%%%%%%%%%%%%%%%%%%%%%%%%%%%%%%%%%%%%%%%%%%%%%%

\documentclass[12pt]{article} 
\usepackage{alphalph}
\usepackage[utf8]{inputenc}
\usepackage[russian,english]{babel}
\usepackage{titling}
\usepackage{amsmath}
\usepackage{graphicx}
\usepackage{enumitem}
\usepackage{amssymb}
\usepackage[super]{nth}
\usepackage{everysel}
\usepackage{ragged2e}
\usepackage{geometry}
\usepackage{fancyhdr}
\usepackage{cancel}
\usepackage{siunitx}
\usepackage{xcolor}
\geometry{top=1.0in,bottom=1.0in,left=1.0in,right=1.0in}
\newcommand{\subtitle}[1]{%
  \posttitle{%
    \par\end{center}
    \begin{center}\large#1\end{center}
    \vskip0.5em}%

}
\usepackage{hyperref}
\hypersetup{
colorlinks=true,
linkcolor=blue,
filecolor=magenta,      
urlcolor=blue,
citecolor=blue,
}

\urlstyle{same}


\title{Lecture 7 Notes}
\date{June 24, 2021}
\author{Michael Brodskiy\\ \small Instructor: Prof. Kropf}

% Mathematical Operations:

% Sum: $$\sum_{n=a}^{b} f(x) $$
% Integral: $$\int_{lower}^{upper} f(x) dx$$
% Limit: $$\lim_{x\to\infty} f(x)$$

\begin{document}

    \maketitle

    \begin{enumerate}

      \item Political Culture Theory — A state-level theory

      \item A state or country (the terms are synonymous) is a political, legal, and territorial entity. A state consists of an internationally recognized territory, a permanent population, and a government that has control over people within its acknowledged boundaries

      \item In theory, states are sovereign. Sovereignty is recognized by international law and by other states through diplomatic relations

      \item On a broad level, the interaction of states is guided by each state's government, which is a public institution that has the authority to create, implement, and enforce rules, laws, and decisions within a state's territorial borders

      \item Governments present themselves and their interests to the international community using foreign policy. Foreign policy consists of decisions and strategies used by governments to guide their interaction with other states in the international system

      \item The regimist perspective emphasizes the nature of a state's government — democratic or authoritarian — as a vital factor in its foreign policy and behavior within the global system

      \item The civilizationist perspective of political culture theory acknowledges some points made by proponents of the regime category, but stress the importance of culture and civilization in determining a state's behavior in IR

      \item Culture refers to a particular group's commonly shared behavior patterns, including language, traditions, values, customs, institutions, and beliefs

      \item Political culture points specifically to the dominant values, attitudes, and beliefs that affect the politics and behavior of individual governments

      \item Political culture theorists — both regimists and civilizationists — argue that a state's political culture has a substantial influence over its foreign policy

      \item Political culture theorists who emphasize the importance of regimes argue that it is the inherent differences between various types of government that can influence state behavior on the world stage and international politics

      \item Regimists

        \begin{enumerate}

          \item Unlike system-level theorists who point to the characteristics of the system as a whole as the central force in international politics — political culture theorists suggest that the domestic characteristics of state governments are key determinants of world politics

          \item Liberal democracies (or “open” societies), where citizens have a voice in government through representatives and political parties, generally take a more pluralistic approach to foreign policy-making

          \item Pluralism describes a political system in which decisions and policies are formulated on the basis of many different viewpoints or interests

          \item Conversely, in authoritarian regimes (or “closed” societies), decisions and policies are made by an individual or small group of leaders. People in these types of regimes generally have no meaningful impact on the political agenda or foreign policy of the country

          \item Under this system of decision making, sometimes called elitism, the policies that control the actions of the state and those who live within it are created and directed by a small ruling elite

        \end{enumerate}

      \item Civilizationists

        \begin{enumerate}

          \item The second perspective on political culture theory focuses on Samuel Huntington’s examination of the impact of broad cultural factors on the behavior of individual states, or even groups of states that share a common culture

          \item Huntington outlined this new theory of international politics for the post-Cold War world in his 1993 \textit{Foreign Affairs} article titled “The Clash of Civilizations”

          \item Huntington defines civilization as the “highest cultural grouping of people and the broadest level of cultural identity people have, short of that which distinguishes them from other species”

          \item “The Clash of Civilizations” lays out a theoretical framework suggesting that cultural differences have overtaken ideological differences as the most important source of conflict among peoples and states

          \item The argument centers on the idea that cultural differences between the eight civilizations – Western, Confucian, Japanese, Islamic, Hindu, Slavic-Orthodox, Latin American, and African – are more serious and could be even more dangerous than the traditional ideological and economic clashes characterizing past wars

          \item A cultural fault line is found where different civilizations share a common border. According to Huntington, cultural fault lines now represent the most probable new “flash points for crisis and bloodshed”

        \end{enumerate}

      \item Components

        \begin{enumerate}

          \item Focus of Analysis — Domestic, political, and cultural characteristics of states or civilizations

          \item Major Influences — Type of government or civilization

          \item Behavior of States — (Regimists) Behavior based on type of government; (Civilizationists) Competitive, based on cultural values

          \item Basis of a State's Foreign Policy — Assumes that a state’s foreign policy reflects the dominant values, attitudes, and beliefs of society/civilizations

        \end{enumerate}

      \item Key Concepts

        \begin{enumerate}

          \item Authoritarian Regimes — Societies in which dominant political authority and power resides in an individual or small group of leaders who are not responsible to the people under their control

          \item Civilization — Composed of elements that bind people together such as common language, religion, customs, institutions, and identification with a particular culture. Huntington defines civilization as the “highest cultural grouping of people and the broadest level of cultural identity people have short of that which distinguishes them from other species”

          \item Civilizationist Perspective — One of two major divisions of political culture theory and is based on the theory set forth by Samuel Huntington in his article “The Clash of Civilizations?” The civilizationist perspective stresses that the “principle conflicts of global politics will occur between nations and groups of different civilizations” 

          \item Cultural Fault Line — The area where different civilizations share a common border; this includes both borders between different states as well as the more tenuous ones found within states. This term was also coined by Samuel Huntington

          \item Culture — Refers to commonly shared behavior patterns, including language, traditions, values, customs, institutions, and religious beliefs of a particular social group

          \item Elitism — A condition where a small group of people control, rule, or dominate the actions of a state. This group formulates both the domestic and foreign policy agendas

          \item End of History — A phrase used by Francis Fukuyama to describe the triumph of western economic and political liberalism over its ideological alternatives, such as fascism and Marxism-Leninism, and the coming universalization of western liberal democracy

          \item Foreign Policy — The strategy used by a government to make decisions and guide its interaction with other states in the international system.  Typically, foreign policy promotes the political, economic, and military interests of the state

          \item Government — A public institution that has the authority to create, implement, and enforce rules, laws, and decisions with a state’s territorial borders.  These rules, laws, and decisions maintain order within society, as well as project and protect the state’s interests abroad

          \item Liberal Democracies — States in which the citizenry has a voice in government through duly elected representatives from two or more political parties

          \item Pluralism — A political system in which decisions and policies are formulated on the basis of many different viewpoints or interests

          \item Political Culture — Refers to the dominant values, attitudes, and beliefs that affect the politics and behavior of individual governments

          \item Regimist Perspective — One of two major divisions of political culture theory. Regimists argue that democracies and authoritarian regimes behave differently in foreign affairs. Their analysis focuses on regime type as a vital factor in determining a state’s foreign policy and behavior within the global system

          \item Sovereignty — A situation in which the domestic government is the supreme authority within a state and does not answer to any outside power. This sovereignty is recognized by international law and by other states through diplomatic relations and often by membership in the United Nations

          \item State — A political, legal, and territorial entity. A state consists of an internationally recognized territory, a permanent population, and a government that has control over the people within its acknowledged boundaries

        \end{enumerate}

      \item Critiques of Political Culture Theory

        \begin{enumerate}

          \item As far as regime theory goes, critics argue democracies can find themselves in conflict, and the so-called “peace among democracies” may be the result of particular geopolitical factors rather than the influence of shared common values

          \item Is civilizational consciousness truly a uniting or dividing force among states? Are cultural, religious, and ethnic factors vitally important sources of states’ actions? Can the principles of liberal democracy bring states and peoples together\dots? \dots or could the actions of states be shaped more by systemic factors, compelling all governments — irrespective of regime or civilization — to act in similar manners?

        \end{enumerate}

      \item Decision Making Process Theory — A state-level theory that emphasizes the role of bureaucratic organizations and how the process of decision-making within these organizations, and within the government as a whole, can affect its foreign policy and international relations

      \item Decision-making process theory disassembles the state bureaucratic structure, taking a targeted look at specific component parts to understand how decisions are made

      \item The idea behind this theory is that different factions of a large bureaucracy are likely to approach problems or issues from different perspectives and with different preconceptions and priorities

      \item So bureaucratic agencies address issues from different points of view, or, as the saying goes, “Where you stand is largely determined by where you sit”

      \item According to decision-making process theorists, the inner workings of the bureaucracy – bargaining strategies, departmental priorities, and the like – cannot be disassociated from an assessment of foreign policy or international relations

      \item A bureaucracy is a network of interconnected departments and organizations designed to manage and administrate the operations of a state; The method that a state or bureaucracy uses to reach decisions is called decision-making 

      \item A key part of this process and a defining feature of large bureaucracies is their dependence on standard operating procedures (SOPs). SOPs are accepted routines or patterns used by bureaucracies to organize and simplify the decision-making process

      \item New policies are often the result of what is called incrementalism, which means that bureaucracies allow only incremental or marginal alterations in existing policy to prevent major changes from established norms

      \item Decision-making process theorists view foreign policy and international relations largely as a by-product of these inner workings and the interaction of various state bureaucracies

      \item Graham T. Allison’s three distinct models for explaining foreign policy decisions:

        \begin{enumerate}

          \item With the rational actor model, decision makers carefully define and identify foreign policy problems, gather all available information about the foreign policy options, weigh all possible alternatives, and select policies that are most likely to promote the state’s national interests

          \item The organizational process model focuses on the routines, standard patterns of behavior, and institutional perspectives of particular agencies, and their impact on foreign policy decisions

          \item The bureaucratic politics model emphasizes the struggle between various agencies of the government and its impact on the decision-making process. This model contends that the formulation of policy is largely the result of the competition among government agencies, representing diverse views

        \end{enumerate}

      \item Components

        \begin{enumerate}

          \item Focus of Analysis — Internal decision-making process of governments

          \item Major Actors — Bureaucracies, agencies, and organizations within government

          \item Behavior of States — States are not necessarily rational actors; Complex interplay of internal and external factors

          \item Basis of a State's Foreign Policy — Compromise/competition between government agencies; Adherence to standard operating procedures; Incrementalism

        \end{enumerate}

      \item Key Concepts

        \begin{enumerate}

          \item Bureaucracy — A network of interconnected departments and organizations designed to manage and administrate the operations of a state

          \item Bureaucratic Politics Model — One of three conceptual models devised by Graham Allison to explain and predict the foreign policy behavior of states. The bureaucratic model focuses on the struggle between various agencies of the government and its impact on the decision-making process. This model contends that the formulation of policy is largely the result of the competition among government agencies, representing diverse views. Such a competitive process means that foreign policy is often based more on domestic political struggles than on objective calculations of the national interest
            
          \item Decision Making — Decision making it the process of identifying problems, devising alternative policy options, and selecting which one of the alternatives to pursue

          \item Incrementalism — A tendency of decision makers to make only incremental only incremental or marginal alterations in existing policy in order to prevent major changes from established norms

          \item Organizational Process Model — The organizational process model is the second of the three models devised by Graham Allison to explain and predict the foreign policy behavior of states. The organizational process model focuses on the routines, standard patterns of behavior, and institutional perspectives of particular agencies and their impact on foreign policy decisions. It assumes that all governments generally rely on standard operating procedures, are relatively predictable, and favor only marginal changes in existing policy

          \item Rational Actor Model — The rational actor model is associated with the realist theory of decision making and the first of three models used by Graham Allison to explain and predict the foreign policy behavior of states. According to the rational actor model, decision makers carefully define and identify foreign policy problems, gather all available information about the foreign policy options, weigh all possible alternatives, and select policies that are most likely to promote the state's national interests

          \item Standard Operating Procedures — Accepted routines or patterns used by bureaucracies to organize and simplify the decision making process. Standard operating procedures are utilized to handle problems or make decisions on a wide array of issues that confront governments on a daily basis—from internal personnel decisions to matters of international trade and diplomacy

        \end{enumerate}

      \item Critiques of Decision-Making Process Theory

        \begin{enumerate}

          \item Realists criticize the inclusive approach of decision-making process theory for focusing too much on the details of foreign policy-making process, thus losing the broader analytical and theoretical implications of history and events

          \item Because they attempt to incorporate such a broad range of factors into their analysis, decision-making process theorists are often accused of describing the interaction of states, as opposed to providing a true story of international relations

          \item Decision making theory has also been attacked for being western- centric (meaning it is only valuable when explaining the foreign policy process of western democratic governments)

        \end{enumerate}

    \end{enumerate}

\end{document}

