%%%%%%%%%%%%%%%%%%%%%%%%%%%%%%%%%%%%%%%%%%%%%%%%%%%%%%%%%%%%%%%%%%%%%%%%%%%%%%%%%%%%%%%%%%%%%%%%%%%%%%%%%%%%%%%%%%%%%%%%%%%%%%%%%%%%%%%%%%%%%%%%%%%%%%%%%%%%%%%%%%%%%%%%%%%%%%%%%%%%%%%%%%%%
% Written By Michael Brodskiy
% Class: International Relations
% Professor: J. Kropf
%%%%%%%%%%%%%%%%%%%%%%%%%%%%%%%%%%%%%%%%%%%%%%%%%%%%%%%%%%%%%%%%%%%%%%%%%%%%%%%%%%%%%%%%%%%%%%%%%%%%%%%%%%%%%%%%%%%%%%%%%%%%%%%%%%%%%%%%%%%%%%%%%%%%%%%%%%%%%%%%%%%%%%%%%%%%%%%%%%%%%%%%%%%%

\documentclass[12pt]{article} 
\usepackage{alphalph}
\usepackage[utf8]{inputenc}
\usepackage[russian,english]{babel}
\usepackage{titling}
\usepackage{amsmath}
\usepackage{graphicx}
\usepackage{enumitem}
\usepackage{amssymb}
\usepackage[super]{nth}
\usepackage{everysel}
\usepackage{ragged2e}
\usepackage{geometry}
\usepackage{fancyhdr}
\usepackage{cancel}
\usepackage{siunitx}
\usepackage{xcolor}
\geometry{top=1.0in,bottom=1.0in,left=1.0in,right=1.0in}
\newcommand{\subtitle}[1]{%
  \posttitle{%
    \par\end{center}
    \begin{center}\large#1\end{center}
    \vskip0.5em}%

}
\usepackage{hyperref}
\hypersetup{
colorlinks=true,
linkcolor=blue,
filecolor=magenta,      
urlcolor=blue,
citecolor=blue,
}

\urlstyle{same}


\title{Lecture 2 Notes}
\date{June 15, 2021}
\author{Michael Brodskiy\\ \small Instructor: Prof. Kropf}

% Mathematical Operations:

% Sum: $$\sum_{n=a}^{b} f(x) $$
% Integral: $$\int_{lower}^{upper} f(x) dx$$
% Limit: $$\lim_{x\to\infty} f(x)$$

\begin{document}

    \maketitle

    \begin{enumerate}

      \item Concepts in IR are shaped by historical circumstances. The state, nation, sovereignty, power, balance of power, and many others are ideas rooted in the European experience

      \item Thucydides

        \begin{enumerate}

          \item His \textit{History of the Peloponnesian War} discusses the causes of the war between Athens and Sparta. His conclusion was that changing distributions of power lead to war

        \end{enumerate}

      \item Plato

        \begin{enumerate}

          \item His \textit{Republic} describes the “perfect state”, where people who govern are those who are superior in the ways of philosophy and war. He introduced two important ideas to IR: class analysis and dialectical reasoning

        \end{enumerate}

      \item Aristotle

        \begin{enumerate}

          \item He is the first to use the comparative method to look at similarities and differences among states. He concluded that states rise and fall due to internal factors — a conclusion still debated in the \nth{21} century

        \end{enumerate}

      \item The Roman Empire

        \begin{enumerate}

          \item Originates the concept of imperialism, and develops the practice of expanding territorial reach. The empire itself is united through law and language, while allowing some local identity

        \end{enumerate}

      \item The early-to-mid Middle Ages

        \begin{enumerate}

          \item During this period, three civilizations emerge from Rome — Arabic, Byzantine, and European. Since European civilization was in a state of disorder, some scholars believe that feudalism arose as a response to disorder

          \item The preeminent institution during this period was the church. Thus, a centralization of religious authority and a decentralization in political and economic life characterizes the era.

        \end{enumerate}

      \item The late Middle Ages

        \begin{enumerate}

          \item A period of rapid economic expansion and exploration. A new group emerges — the business community, whose interests extend beyond their immediate locales and who conflict with the church

          \item This is the era of Machiavelli. In \textit{The Prince}, he points out the necessary qualities of a leader required to maintain the strength and security of the state

        \end{enumerate}

      \item The Emergence of the Westphalian System

        \begin{enumerate}

          \item Begins upon completion of the Thirty Years' War (1618$-$1648), one of the worst wars (religious in character) in human history with battles that ravaged much of the civilian population

          \item The treaty that followed (Treaty of Westphalia) impacted IR in 3 ways:

            \begin{enumerate}

              \item It created the concept of sovereignty, or the “authority of the state, based on recognition by other states and non-state actors, to govern matters within its own borders that affect its people, economy, security, and form of government.”

              \item It saw the formation of national armies which further centralized control

              \item It established a core group of states that dominated the world until the beginning of the \nth{19} century (Austria, Russia, Prussia, England, France, and the United Provinces)

            \end{enumerate}

        \end{enumerate}

      \item The \nth{19} Century in Europe

        \begin{enumerate}

          \item Dominated by 2 revolutions, the American (1776) and French (1789), from which 2 core principles emerged:

            \begin{enumerate}

              \item Legitimacy — Absolutist rule is subject to limits imposed by man

              \item Nationalism — The masses identify with their common past, language, customs, and practices as a natural outgrowth of the state

            \end{enumerate}

          \item The Concert of Europe (1815$-$1854) was a period of relative peace in the international system, even though great political changes were occurring. This was due to:

            \begin{enumerate}

              \item Solidarity due to their shared European, Christian, “civilized” and “white” background, which differentiated between “them” and the “other”

              \item European elites united in their fear of revolution from the masses

              \item Industrialization and focus on colonialism

              \item The Balance of Power concept — with each relatively equal in power, they feared emergence of any predominant state (hegemon) among them

            \end{enumerate}

        \end{enumerate}

      \item World War I — The Breakdown of Balance of Power

        \begin{enumerate}

          \item The end of the war denotes critical changes in international relations

          \item Three European empires die — Russia, Austria-Hungary, and Ottoman

          \item Germany emerges as even more dissatisfied

          \item Enforcement of the Treaty of Versailles was given to the League of Nations. It fails (no power, no legal instruments, no legitimacy). The US refuses to join, which creates a unilateralist foreign policy.

          \item The rise of fascism — German, Italian, Japanese, and the resurgence of nationalisms

          \item The combination of a world economic decline, along with fascism, liberalism, and communism clashing leads to World War II

        \end{enumerate}

      \item World War II and The Cold War

        \begin{enumerate}

          \item The most important outcome of World War II was the emergence of two superpowers — The US and the Soviet Union, and the decline of Europe as the center of international politics

          \item Related to the first outcome was the recognition of the fundamental incompatibilities between these two superpowers in national interest and ideology, particularly the ideologies of capitalism and socialism

          \item The third outcome was the realization that, because of nuclear power, the differences between the US and USSR would be played out indirectly, on third-party stages, rather than direct confrontation. It was through this “globalized” conflict that international relations became truly international

        \end{enumerate}

      \item The Cold War as the “Long Peace”

        \begin{enumerate}

          \item Just as general war was avoided in \nth{19} century Europe, it has also been avoided since World War II. Gaddis suggests 5 factors:

            \begin{enumerate}

              \item Nuclear “deterrence”

              \item Bipolarity (equality of power)

              \item Hegemonic economic power of the US

              \item Pluralism/liberalism/transnationalism

              \item Historical cycles (global wars every 100-150 years)

            \end{enumerate}

        \end{enumerate}

      \item Key Developments in the post-Cold War Era

        \begin{enumerate}

          \item Changes in Russian foreign policy

          \item Iraqi invasion of Kuwait and multilateral response unites former Cold War adversaries

          \item Disintegration of former Yugoslavia into independent states; civil wars in Bosnia; NATO action in Serbia

          \item September 11, 2001 and the global “War on Terror”

        \end{enumerate}

    \end{enumerate}

\end{document}

