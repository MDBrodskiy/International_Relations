%%%%%%%%%%%%%%%%%%%%%%%%%%%%%%%%%%%%%%%%%%%%%%%%%%%%%%%%%%%%%%%%%%%%%%%%%%%%%%%%%%%%%%%%%%%%%%%%%%%%%%%%%%%%%%%%%%%%%%%%%%%%%%%%%%%%%%%%%%%%%%%%%%%%%%%%%%%%%%%%%%%%%%%%%%%%%%%%%%%%%%%%%%%%
% Written By Michael Brodskiy
% Class: International Relations
% Professor: J. Kropf
%%%%%%%%%%%%%%%%%%%%%%%%%%%%%%%%%%%%%%%%%%%%%%%%%%%%%%%%%%%%%%%%%%%%%%%%%%%%%%%%%%%%%%%%%%%%%%%%%%%%%%%%%%%%%%%%%%%%%%%%%%%%%%%%%%%%%%%%%%%%%%%%%%%%%%%%%%%%%%%%%%%%%%%%%%%%%%%%%%%%%%%%%%%%

\documentclass[12pt]{article} 
\usepackage{alphalph}
\usepackage[utf8]{inputenc}
\usepackage[russian,english]{babel}
\usepackage{titling}
\usepackage{amsmath}
\usepackage{graphicx}
\usepackage{enumitem}
\usepackage{amssymb}
\usepackage[super]{nth}
\usepackage{everysel}
\usepackage{ragged2e}
\usepackage{geometry}
\usepackage{fancyhdr}
\usepackage{cancel}
\usepackage{siunitx}
\usepackage{xcolor}
\geometry{top=1.0in,bottom=1.0in,left=1.0in,right=1.0in}
\newcommand{\subtitle}[1]{%
  \posttitle{%
    \par\end{center}
    \begin{center}\large#1\end{center}
    \vskip0.5em}%

}
\usepackage{hyperref}
\hypersetup{
colorlinks=true,
linkcolor=blue,
filecolor=magenta,      
urlcolor=blue,
citecolor=blue,
}

\urlstyle{same}


\title{Lecture 8 Notes}
\date{June 29, 2021}
\author{Michael Brodskiy\\ \small Instructor: Prof. Kropf}

% Mathematical Operations:

% Sum: $$\sum_{n=a}^{b} f(x) $$
% Integral: $$\int_{lower}^{upper} f(x) dx$$
% Limit: $$\lim_{x\to\infty} f(x)$$

\begin{document}

    \maketitle

    \begin{enumerate}

      \item Human Nature Theory — Human nature refers to the qualities and traits shared by all people, regardless of ethnicity, gender, and culture. Human nature theorists assert that the behavior of states is fundamentally patterned after the behavior of humans themselves

      \item Human nature theorists argue that, on a basic level, human nature and the nature of other species as well, is often guided by instinct

      \item This is instinct and definitely "aggressive" according to Lorenz. However! Is this "intraspecific" or "interspecific" aggression? 

      \item Instinct could be defined as an innate impulse that is prompted in response to specific environmental conditions.

      \item Aristotle saw the nature of man as essentially divided into two parts (good and evil) and therefore saw life and relations between people as also divided

      \item Aristotle believed that the human race can emphasize one facet over the other (peace over war), but this does not negate the presence of the less desirable side of existence

      \item Thomas Hobbes in his classic work Leviathan, asked us to imagine the wretchedness of human existence in a “state of nature” when there was no central government able to control the baser instincts of humans and to provide order

      \item Hobbes points to three specific causes of conflict that are endemic to human nature: 1. Competition 2. Diffidence 3. Glory 

      \item Competition represents the perpetual struggle between humans for resources, power, or anything else that might represent some sort of gain

      \item Diffidence arises from a perception of equality or a sense of self-satisfaction that occurs when individuals attain a particular set of goals or ends

      \item The quest for glory in human nature is the third portion of Hobbes’ trilogy about the unavoidable violence of existence. He defines glory as the quest for a preservation of honor, respect, or reputation as it refers to the individual, or as it reflects on the family, friends, name, associated with the individual.

      \item Sigmund Freud argued that the natural violence of the human animal can actually be “overcome by the transference of power to a larger unity” or group. In his letter to Albert Einstein addressing the scientist’s question about how future wars could be avoided, Freud stated that transferring power to a central authority reduces the possibility of violence

      \item Despite Freud’s hopeful acknowledgment, the enduring theme for human nature theorists is that humans are by nature – and can be generally counted on to be – violent and aggressive. Human nature theorists suggest that these traits, so instinctive to men and women, are thereby carried over to our relations on an international level

      \item Components

        \begin{enumerate}

          \item Focus of Analysis — Human nature; Innate patterns of human behavior; Instinct

          \item Major Actors — The individual

          \item Behavior of States — Guided by fundamental human characteristics

          \item Basis of a State's Foreign Policy — Collective protection of individual against more violent tendencies of human nature

        \end{enumerate}

      \item Key Concepts

        \begin{enumerate}

          \item Competition — A term used to represent the perpetual struggle between humans for resources, power, or anything else that might represent some sort of gain. According to Thomas Hobbes, competition is one of the three specific causes of conflict that are endemic to human nature.  The other two causes are diffidence and glory

          \item Diffidence — One of the three causes of conflict, according to Thomas Hobbes, that are endemic to human nature. Diffidence arises from a human's perception of equality or a sense of self-satisfaction and occurs when individuals attain a particular set of goals or ends

          \item Glory — Defined by Thomas Hobbes as the quest for and preservation of honor, respect, or reputation as it refers to the individual or as it reflects on the family, friends, name, and so forth, associated with that individual. According to Hobbes, glory is one of the three causes of conflict that are endemic to human nature

          \item Human Nature — Human nature refers to the qualities and traits shared by all people, regardless of ethnicity, gender, culture, etc. Human nature theorists assert that the behavior of states is fundamentally patterned after the behavior of humans themselves. Human nature, then, can provide clues about when and why states might behave in a particular fashion

          \item Instinct — An innate pattern of behavior characteristic of species, including humans. To varying degrees, humans and other species are preprogrammed to respond in a certain manner when confronted with a particular set of circumstances

          \item State of Nature — State of nature refers to Thomas Hobbes' pessimistic view of life prior to the creation of a central government or authority to control the baser instincts of humans and provide order. Under these conditions the life of humans would be "solitary, poore, nasty, brutish, and short"  

        \end{enumerate}

      \item Critiques of Human Nature Theory

        \begin{enumerate}

          \item Critics of human nature theory argue that it does not delve deeply enough into the driving forces of foreign policy and global politics. For instance, if we assume that the human animal is by nature competitive and prone to violence, this leads us to several questions including: 

            \begin{enumerate}

              \item First, does human nature theory provide a sufficient explanation of the complex character of international relations? Does it have the analytical depth necessary to understand IR in our world today?

              \item Also, can we assume that human beings’ natural aggression and lust for power or even Aristotle’s notion of the conflict between good and evil in individuals, represent the driving forces behind not only the behavior of states but behind all of these other events, as well?

              \item Finally, since “we” don’t decide whether or not to go to war with another nation, should be take a closer look at individual leaders, and their particular features, and unique ambitions? 

            \end{enumerate}

        \end{enumerate}

      \item Cognitive Theory — Unlike human nature theory which posits that all humans share innate traits that lead to conflict, cognitive theorists suggest that it is a leader’s specific personality that guides not only his or her own actions but the destiny of the state and its relations with other countries 

      \item We can define personality as the package of behavior, temperament, and other individualistic qualities that uniquely identifies each of us

      \item Cognition is one element of this personality package.  Cognition is what an individual comes to know as a result of learning and reasoning or, on a more instinctual level, intuition and perception

      \item Cognitive theorists believe that the personality traits of leaders can often define both the agenda and specific features of a state’s foreign policy

      \item With this theory, a leader’s individual perception of reality is naturally conditioned by emotional attachments and aversions that he or she has formed in life

      \item Margaret Hermann provides a structure for analyzing the individual traits that might affect a leader’s decisions and policy making. She concludes that leaders generally have one of two orientations toward foreign affairs: 1. Independent 2. Participatory 

      \item An independent leader tends to be aggressive, with a limited capacity to consider different alternatives, and is willing to be the first to take action over a perceived threat

      \item A participatory leader is generally conciliatory, with an inclusive nature that encourages relationships with other countries, considers various alternatives in problem solving, and rarely seeks to initiate action

      \item Cognitive theorists argue that a leader’s perception of policy, the state, and the system will be conditioned by these same features.  Operational reality refers to the picture of the environment held by an individual (usually a leader) as it is modified by his or her personality, perceptions, and misperceptions

      \item Components

        \begin{enumerate}

          \item Focus of Analysis — Personality and cognitive experience of leaders

          \item Major Actors — Individuals

          \item Behavior of States — Guided by experience, preconceptions, background, and personality of leader(s)

          \item Basis of a State's Foreign Policy — Driven by operational reality of elites

        \end{enumerate}

      \item Key Concepts

        \begin{enumerate}

          \item Cognition — An individuals knowing as a result of learning and reasoning or, on a more instinctual level, intuiting and perceiving. All of these components combine to form a person's cognitive facility

          \item Independent Leader — One of the two major orientations of leaders (Hermann).  They tend to be aggressive, have a limited capacity to consider different alternatives, and are willing to be the first to take action over a perceived threat

          \item Operational Reality — Refers to the picture of the environment held by an individual as it is modified by his or her personality, perceptions, and misperceptions

          \item Participatory Leader

          \item Personality

        \end{enumerate}

      \item Critiques of Cognitive Theory 

        \begin{enumerate}

          \item Critics of cognitive theory might argue that focusing on personality profiles of individual leaders to determine an analytical framework for studying international relations is limiting

          \item One problem: Assessment of any leader should be based on a sophisticated and complete psychoanalysis. Such a scenario is improbable since most leaders have not subjected and would not likely subject themselves to this type of scrutiny 

          \item There is also the question of scope. The emphasis of the theory (personality/cognition) is so narrow that it seems inappropriate to attempt a broad, all-encompassing application to international relations as a whole

          \item In addition, the principles of cognitive theory might lead us to believe that all wars are simply the result of misperceptions or misunderstandings between individual leaders

          \item Cognitive theorists ignore the fact that wars often result from fundamental conflicting interests between states on larger issues, such as national security, competition over scarce resources, or dozens of other factors that affect global politics

        \end{enumerate}

    \end{enumerate}

\end{document}

