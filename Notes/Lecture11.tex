%%%%%%%%%%%%%%%%%%%%%%%%%%%%%%%%%%%%%%%%%%%%%%%%%%%%%%%%%%%%%%%%%%%%%%%%%%%%%%%%%%%%%%%%%%%%%%%%%%%%%%%%%%%%%%%%%%%%%%%%%%%%%%%%%%%%%%%%%%%%%%%%%%%%%%%%%%%%%%%%%%%%%%%%%%%%%%%%%%%%%%%%%%%%
% Written By Michael Brodskiy
% Class: International Relations
% Professor: J. Kropf
%%%%%%%%%%%%%%%%%%%%%%%%%%%%%%%%%%%%%%%%%%%%%%%%%%%%%%%%%%%%%%%%%%%%%%%%%%%%%%%%%%%%%%%%%%%%%%%%%%%%%%%%%%%%%%%%%%%%%%%%%%%%%%%%%%%%%%%%%%%%%%%%%%%%%%%%%%%%%%%%%%%%%%%%%%%%%%%%%%%%%%%%%%%%

\documentclass[12pt]{article} 
\usepackage{alphalph}
\usepackage[utf8]{inputenc}
\usepackage[russian,english]{babel}
\usepackage{titling}
\usepackage{amsmath}
\usepackage{graphicx}
\usepackage{enumitem}
\usepackage{amssymb}
\usepackage[super]{nth}
\usepackage{everysel}
\usepackage{ragged2e}
\usepackage{geometry}
\usepackage{fancyhdr}
\usepackage{cancel}
\usepackage{siunitx}
\usepackage{xcolor}
\geometry{top=1.0in,bottom=1.0in,left=1.0in,right=1.0in}
\newcommand{\subtitle}[1]{%
  \posttitle{%
    \par\end{center}
    \begin{center}\large#1\end{center}
    \vskip0.5em}%

}
\usepackage{hyperref}
\hypersetup{
colorlinks=true,
linkcolor=blue,
filecolor=magenta,      
urlcolor=blue,
citecolor=blue,
}

\urlstyle{same}


\title{Lecture 11 Notes}
\date{July 8, 2021}
\author{Michael Brodskiy\\ \small Instructor: Prof. Kropf}

% Mathematical Operations:

% Sum: $$\sum_{n=a}^{b} f(x) $$
% Integral: $$\int_{lower}^{upper} f(x) dx$$
% Limit: $$\lim_{x\to\infty} f(x)$$

\begin{document}

    \maketitle

    \begin{enumerate}

      \item Globalization — Refers to the expansion of global linkages, the organization of social life on a global scale, and the growth of a global consciousness, hence to the consolidation of world society
        
      \item Such a definition captures much of what the term commonly means, but its meaning is disputed. It encompasses several large processes; definitions differ in what they emphasize

      \item Globalization is historically complex; definitions vary in the particular driving force they identify. The meaning of the term is itself a topic in global discussion; it may refer to “real” processes, to ideas that justify them, or to a way of thinking about them

      \item The term is not neutral!  Definitions express different assessments of global change. Among critics of capitalism and global inequality, globalization now has an especially pejorative ring

      \item The following definitions represent currently influential views from academia:

        \begin{enumerate}

          \item Friedman, \textit{The Lexus and the Olive Tree}, 1999 — “[T]he inexorable integration of markets, nation-states, and technologies to a degree never witnessed before in a way that is enabling individuals, corporations and nation-states to reach around the world farther, faster, deeper and cheaper than ever before \dots the spread of free-market capitalism to virtually every country in the world” 

          \item Robertson, \textit{Globalization}, 1992 — “The compression of the world and the intensification of consciousness of the world as a whole \dots concrete global interdependence and consciousness of the global whole in the twentieth century”

          \item Waters, \textit{Globalization}, 1995 — “A social process in which the constraints of geography on social and cultural arrangements recede and in which people become increasingly aware that they are receding” 

          \item Albrow, \textit{The Global Age}, 1996 — “The historical transformation constituted by the sum of particular forms and instances of \dots [m]aking or being made global (i) by the active dissemination of practices, values, technology and other human products throughout the globe (ii) when global practices and so on exercise an increasing influence over people’s lives (iii) when the globe serves as a focus for, or a premise in shaping human activities”

          \item McMichael, \textit{Development and Social Change}, 2000 — “Integration on the basis of a project pursuing ‘market rule on a global scale’”

          \item Mittelman, \textit{The Globalization Syndrome}, 2000 — “As experienced from below, the dominant form of globalization means a historical transformation: in the economy, of livelihoods and modes of existence; in politics a loss in the degree of control exercised locally…and in culture, a devaluation of a collectivity’s achievements… Globalization is emerging as a political response to the expansion of market power… [I]t is a domain of knowledge”

        \end{enumerate}
        
      \item The 4 Types of Globalization:

        \begin{enumerate}

          \item Economic — Involves long-distance flow of goods, services and capital, as well as the information and perceptions that accompany market exchange

            \begin{enumerate}

              \item Will economic globalization contribute to, or serve to ameliorate, poverty and inequality? (The so-called “promise” of neoliberal globalization)?

            \end{enumerate}

          \item Military — Refers to long-distance networks of interdependence in which force and the threat of promise of force, are employed (ex. Nuclear power)

            \begin{enumerate}
                
              \item Will military globalization increase or decrease fear and insecurity? (The peace-through- strength trope)?

            \end{enumerate}

          \item Environmental — Refers to long-distance transport of materials in the atmosphere of the oceans, or of biological substances such as pathogens or genetic materials, that affect human health and well-being

            \begin{enumerate}

              \item Will environmental globalization improve or degrade global environmental conditions with regard to climate, air/water/soil pollution, and continued loss of biodiversity?

            \end{enumerate}

          \item Social and Cultural — Involves the movement of ideas, images, and people; religious and scientific knowledge; and the imitation of one society’s practices and institutions by others – which sociologists refer to as “isomorphism” 

            \begin{enumerate}

              \item Will socio-cultural globalization work to bring diverse peoples together to solve global problems, or accentuate differences leading to more fundamentalism and more terrorism/violence?

            \end{enumerate}

        \end{enumerate}

      \item Stage 1 (Pre-1492) — The First “Globalizers”

        \begin{enumerate}

          \item Prior to Columbus, most economic activity was local

          \item Armies were the entities that covered vast distances in pursuit of conquest

          \item Vikings and Marco Polo are examples of pre-Colombian “globalizers” who brought with them exotic foods, spices, crafts, etc\dots

          \item Chinese and Arab traders moved goods across Asia, the Middle East, and North Africa

        \end{enumerate}

        \item Stage 2 (1492$-$1945) — Empire and a Colonial Division of Labor

        \begin{enumerate}

          \item Beginning in the fifteenth century, European powers financed explorations to Africa, Asia, and the Western Hemisphere. Soon these explorations turned to conquest and a “colonial division of labor”

          \item Foreign lands produce the raw materials and the “mother country” produces the manufactured products (a price scissors)

          \item Slave trade expands in North America

        \end{enumerate}

      \item Stage 3 (1945$-$2018) — New Divisions

        \begin{enumerate}

          \item The pre-1945 divisions of labor have changed radically in the last few decades. While the richer countries of Europe and North America along with Japan still largely export industrial products, among the poorer nations, six groups have emerged:

            \begin{enumerate}

              \item Big Emerging Markets — Mexico, Brazil, Argentina, South Africa, China, India, Indonesia, Korea, Turkey.  These countries are still relatively poor

              \item Would-be BEMs — Colombia, Venezuela, Chile, Greece, Portugal, Thailand, and Malaysia. They have moved beyond clothing and electronics into more diversified industrial and service sectors

              \item OPEC Nations — These are oil-exporting countries

              \item Former Communist Economies — despite a relatively high state of industrialization, most of the 26 former Soviet Bloc nations are struggling and finding it difficult to compete

              \item Raw Materials Exporters and Light Manufacturers — About 40 countries have little heavy industry and are suppliers of raw materials

              \item Least Developed Countries — About 60 countries, mostly in Africa, are so poor that their economic connection with the rest of the world is limited to minimal trade and investment and dwindling foreign aid

            \end{enumerate}

        \end{enumerate}

    \end{enumerate}

\end{document}

