%%%%%%%%%%%%%%%%%%%%%%%%%%%%%%%%%%%%%%%%%%%%%%%%%%%%%%%%%%%%%%%%%%%%%%%%%%%%%%%%%%%%%%%%%%%%%%%%%%%%%%%%%%%%%%%%%%%%%%%%%%%%%%%%%%%%%%%%%%%%%%%%%%%%%%%%%%%%%%%%%%%%%%%%%%%%%%%%%%%%%%%%%%%%
% Written By Michael Brodskiy
% Class: International Relations
% Professor: J. Kropf
%%%%%%%%%%%%%%%%%%%%%%%%%%%%%%%%%%%%%%%%%%%%%%%%%%%%%%%%%%%%%%%%%%%%%%%%%%%%%%%%%%%%%%%%%%%%%%%%%%%%%%%%%%%%%%%%%%%%%%%%%%%%%%%%%%%%%%%%%%%%%%%%%%%%%%%%%%%%%%%%%%%%%%%%%%%%%%%%%%%%%%%%%%%%

\documentclass[12pt]{article} 
\usepackage{alphalph}
\usepackage[utf8]{inputenc}
\usepackage[russian,english]{babel}
\usepackage{titling}
\usepackage{amsmath}
\usepackage{graphicx}
\usepackage{enumitem}
\usepackage{amssymb}
\usepackage[super]{nth}
\usepackage{everysel}
\usepackage{ragged2e}
\usepackage{geometry}
\usepackage{fancyhdr}
\usepackage{cancel}
\usepackage{siunitx}
\usepackage{expl3}
\usepackage[version=4]{mhchem}
\usepackage{hpstatement}
\usepackage{xcolor}
\geometry{top=1.0in,bottom=1.0in,left=1.0in,right=1.0in}
\newcommand{\subtitle}[1]{%
  \posttitle{%
    \par\end{center}
    \begin{center}\large#1\end{center}
    \vskip0.5em}%

}
\usepackage{hyperref}
\hypersetup{
colorlinks=true,
linkcolor=blue,
filecolor=magenta,      
urlcolor=blue,
citecolor=blue,
}

\urlstyle{same}


\title{Lecture 14 Notes}
\date{July 15, 2021}
\author{Michael Brodskiy\\ \small Instructor: Prof. Kropf}

% Mathematical Operations:

% Sum: $$\sum_{n=a}^{b} f(x) $$
% Integral: $$\int_{lower}^{upper} f(x) dx$$
% Limit: $$\lim_{x\to\infty} f(x)$$

\begin{document}

    \maketitle

    \begin{enumerate}

      \item Nationalism is an ideology (belief system) which claims supreme loyalty from individuals for the nation and the state that represents it

      \item Nationalism is one of the most important factors in world history and continues to be in contemporary international politics

      \item When did it begin? The political focus on nationalism has evolved over the last five centuries (recall the Thirty Years War and the Treaty of Westphalia)

      \item Where is it going? After World War II, some predicted an end to nationalism. Today, it is stronger than it has ever been

      \item Is nationalism good or bad? Well, nationalism has both positive and negative aspects

      \item On the positive side, it can promote democracy, self-government, economic growth, and social/political/economic diversity and experimentation

      \item On the negative side, it can lead to isolationism, feelings of superiority, suspicion of others, and messianism 

      \item It can also cause instability and lead to foreign intervention and hyper-factionalization of states

      \item Religion in International Politics

        \begin{enumerate}

          \item Religion is one of the most ancient forces that influence world events.  Objectively, it can be said to play a dual role in world politics

          \item On the one hand, it has been a source of humanitarian concern and a vehicle for pacifism, including but not limited to Gandhi and Indian foreign policy, Christian, Islamic, and other religious denominations and their relief work all over the world and in many cases is the basis for the anti-nuclear movement in Europe and the foundation of Liberation Theology in Latin America

          \item However, it has also been a force at the center of many bloody wars, including:

            \begin{enumerate}

              \item The reaction in Europe to Islam, which led to the Crusades (1095$-$1291) 

              \item The Protestant Reformation (1517), which led to the Thirty Years War (1618$-$1648)

            \end{enumerate}

          \item In Western civilization, the process of secularization has led to a separation of church and state. Not so within the Islamic and Hindu world

          \item And religion has also created divisions within countries, such as:

            \begin{enumerate}

              \item The Catholic and Protestant division in Ireland

              \item The Hindu, Muslim, Sikh divisions in India

              \item Divisions between Sunni and Shi’ite Muslims in Iraq

              \item Jewish, Islamic, and Christian divisions in Lebanon and Israel

            \end{enumerate}

        \end{enumerate}

      \item Islamic Concepts and Definitions

        \begin{enumerate}

          \item Islam — (which means “submission to God”, a Muslim being “one who submits”), is a monotheistic religion founded by Muhammad (570$-$632 a.d.), a prophet who received Allah’s teachings in a vision

          \item The Koran — Is the central religious text of Islam, which Muslims consider the verbatim word of God

          \item The Caliphs — Successors to Muhammad. They are described in the Koran as representatives of Allah on Earth, and also as leaders of the Ummah, the spiritual, cultural, and political community of Muslims

          \item The notion of "houses" or "divisions" of the world in Islam such as Dar al-Islam (House of Islam) and Dar al-Harb (House of War) does not appear in the Koran or the Hadiths. This geo-political house of divisions was more acutely framed by a \nth{13} century Islamic scholar, Ibn Taymiyyah, in response to Mongol invasions of Muslim lands

          \item The concept of Jihad, is often in the West translated as “holy war,” but carries the broader idea of “struggle” which could be personal or religious

        \end{enumerate}

      \item Political Heritage of Muslims 3 Historic Elements

        \begin{enumerate}

          \item A triumphant past

          \item A clash with Christian powers, especially European

          \item Domination of Muslims by Others:

            \begin{enumerate}

              \item Defeats after 1500 a.d.

              \item Ottoman Empire after WWI

              \item British/French/American colonialism

            \end{enumerate}

        \end{enumerate}

      \item Islam and Nationalism

        \begin{enumerate}

          \item Today, the Ummah takes the form of a “Muslim pride” movement. This includes a rejection of direct interference from outside powers to the resurrection of cultural traditions, such as:

            \begin{enumerate}

              \item Banning alcohol

              \item Women covering their faces

              \item A legal system based on the Shari'ah

            \end{enumerate}

          \item However, the creation of a united Ummah is not likely in the foreseeable future, primarily because of:

            \begin{enumerate}

              \item Nationalism

              \item Ethnic differences (Iranians, Kazakhs, Pakistanis, and many others who are not Arabic)

              \item Sectarian divisions (Sunni-Shia)

            \end{enumerate}

          \item Within the House of Islam, religious differences are a source of intense conflict

          \item Majority Sunnis and minority Shi’ites are at odds over the proper leadership of the Ummah

          \item Sunnis recognize Abu-Bakr (Muhammad’s close companion and advisor) as the legitimate heir, while Shi’ites recognize Ali (Muhammad’s first cousin and son-in-law)

        \end{enumerate}

    \end{enumerate}

\end{document}

