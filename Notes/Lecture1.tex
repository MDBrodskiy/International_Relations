%%%%%%%%%%%%%%%%%%%%%%%%%%%%%%%%%%%%%%%%%%%%%%%%%%%%%%%%%%%%%%%%%%%%%%%%%%%%%%%%%%%%%%%%%%%%%%%%%%%%%%%%%%%%%%%%%%%%%%%%%%%%%%%%%%%%%%%%%%%%%%%%%%%%%%%%%%%%%%%%%%%%%%%%%%%%%%%%%%%%%%%%%%%%
% Written By Michael Brodskiy
% Class: International Relations
% Professor: J. Kropf
%%%%%%%%%%%%%%%%%%%%%%%%%%%%%%%%%%%%%%%%%%%%%%%%%%%%%%%%%%%%%%%%%%%%%%%%%%%%%%%%%%%%%%%%%%%%%%%%%%%%%%%%%%%%%%%%%%%%%%%%%%%%%%%%%%%%%%%%%%%%%%%%%%%%%%%%%%%%%%%%%%%%%%%%%%%%%%%%%%%%%%%%%%%%

\documentclass[12pt]{article} 
\usepackage{alphalph}
\usepackage[utf8]{inputenc}
\usepackage[russian,english]{babel}
\usepackage{titling}
\usepackage{amsmath}
\usepackage{graphicx}
\usepackage{enumitem}
\usepackage{amssymb}
\usepackage[super]{nth}
\usepackage{everysel}
\usepackage{ragged2e}
\usepackage{geometry}
\usepackage{fancyhdr}
\usepackage{cancel}
\usepackage{siunitx}
\usepackage{xcolor}
\geometry{top=1.0in,bottom=1.0in,left=1.0in,right=1.0in}
\newcommand{\subtitle}[1]{%
  \posttitle{%
    \par\end{center}
    \begin{center}\large#1\end{center}
    \vskip0.5em}%

}
\usepackage{hyperref}
\hypersetup{
colorlinks=true,
linkcolor=blue,
filecolor=magenta,      
urlcolor=blue,
citecolor=blue,
}

\urlstyle{same}


\title{Lecture 1 Notes}
\date{June 15, 2021}
\author{Michael Brodskiy\\ \small Instructor: Prof. Kropf}

% Mathematical Operations:

% Sum: $$\sum_{n=a}^{b} f(x) $$
% Integral: $$\int_{lower}^{upper} f(x) dx$$
% Limit: $$\lim_{x\to\infty} f(x)$$

\begin{document}

    \maketitle

    \begin{enumerate}

      \item \textbf{International Relations} — Concerns relations between different actors in the world, the characteristics of those relations, and their consequences

        \begin{enumerate}

          \item Has to do with the nature of those actors, how they have changed over time, and how their interactions have changed over time

          \item Includes questions of international conflict, economics, and questions that transcend actors but confront them nevertheless

        \end{enumerate}

      \item \textbf{International Relations Theory} — A set of principles and guidelines used to analyze both world events and relations between states. These theories help to assess past and present conditions, and, in turn, a basis for predicting future trends

      \item Key Concepts:

        \begin{enumerate}

          \item Hypothesis — An educated guess or proposition about how or why something, an event or specific set of conditions, occurred

          \item Concept — An idea, though or notion. A universal descriptive word

          \item Generalization — A set of concepts that are related

          \item Theory — A set of generalizations that try to analyze, explain, or predict something

        \end{enumerate}

      \item Descriptive “Empirical” Theories — Try to explain events and circumstances. They are based on description and evaluation of past events, conditions, and patterns of behavior

      \item Prescriptive “Normative” Theories — A set of principles and guidelines that contain value judgments about how the world ought to be, rather than who the world really is. Also known as “normative” theories

      \item Levels of Analysis — A method for examining international relations theory based on three different perspectives or levels (individual, state, and system levels)

        \begin{enumerate}

          \item Individual — An approach that focuses on the role and impact of particular individuals, or looks for explanations based on “human nature” or common characteristics of all individuals

          \item State — An analytical approach that focuses on the domestic or internal causes of state actions. An attempt to explain international relations by emphasizing the internal workings of the state or civilization itself

          \item System — An approach that focuses on the manner in which the structure of the international system shapes and constrains the actions of states

        \end{enumerate}

    \end{enumerate}

\end{document}

